\documentclass{amsart}

%\documentclass[10 pt]{amsart}

\usepackage[ocgcolorlinks,linktoc=all]{hyperref}
\hypersetup{citecolor=blue,linkcolor=red}
\usepackage[parfill]{parskip}
\usepackage{graphicx}

%\usepackage{amsthm}
\usepackage{cleveref}
\crefname{lemma}{Lemma}{Lemmata}
\crefname{equation}{equation}{equations}

\newtheorem{theorem}{Theorem}
\newtheorem{lemma}[theorem]{Lemma}
\newtheorem{proposition}[theorem]{Proposition}
\newtheorem{corollary}[theorem]{Corollary}

\newtheorem*{thmA}{Theorem}
\newtheorem*{thmB}{Theorem}
\newtheorem*{rem}{Remark}
\newtheorem*{thmmain}{Theorem}
\newtheorem*{propmain}{Proposition}

\theoremstyle{definition}
\newtheorem{definition}[theorem]{Definition}
\newtheorem{example}[theorem]{Example}
\newtheorem{xca}[theorem]{Exercise}

\theoremstyle{remark}
\newtheorem{remark}[theorem]{Remark}

\numberwithin{equation}{section}

%Symbols
\renewcommand{\~}{\tilde}
\renewcommand{\-}{\bar}
\newcommand{\bs}{\backslash}
\newcommand{\cn}{\colon}
\newcommand{\sub}{\subset}

\newcommand{\N}{\mathbb{N}}
\newcommand{\R}{\mathbb{R}}
\newcommand{\Z}{\mathbb{Z}}
\renewcommand{\S}{\mathbb{S}}
\renewcommand{\H}{\mathbb{H}}
\newcommand{\C}{\mathbb{C}}
\newcommand{\K}{\mathbb{K}}
\newcommand{\Di}{\mathbb{D}}
\newcommand{\B}{\mathbb{B}}
\newcommand{\8}{\infty}

%Greek letters
\renewcommand{\a}{\alpha}
\renewcommand{\b}{\beta}
\newcommand{\g}{\gamma}
\renewcommand{\d}{\delta}
\newcommand{\e}{\epsilon}
\renewcommand{\k}{\kappa}
\renewcommand{\l}{\lambda}
\renewcommand{\o}{\omega}
\renewcommand{\t}{\theta}
\newcommand{\s}{\sigma}
\newcommand{\p}{\varphi}
\newcommand{\z}{\zeta}
\newcommand{\vt}{\vartheta}
\renewcommand{\O}{\Omega}
\newcommand{\D}{\Delta}
\newcommand{\G}{\Gamma}
\newcommand{\T}{\Theta}
\renewcommand{\L}{\Lambda}

%Mathematical operators
\newcommand{\INT}{\int_{\O}}
\newcommand{\DINT}{\int_{\d\O}}
\newcommand{\Int}{\int_{-\infty}^{\infty}}
\newcommand{\del}{\partial}

\newcommand{\inpr}[2]{\left\langle #1,#2 \right\rangle}
\newcommand{\fr}[2]{\frac{#1}{#2}}
\newcommand{\x}{\times}

\DeclareMathOperator{\dive}{div}
\DeclareMathOperator{\id}{id}
\DeclareMathOperator{\pr}{pr}
\DeclareMathOperator{\Diff}{Diff}
\DeclareMathOperator{\supp}{supp}
\DeclareMathOperator{\graph}{graph}
\DeclareMathOperator{\osc}{osc}
\DeclareMathOperator{\const}{const}
\DeclareMathOperator{\dist}{dist}
\DeclareMathOperator{\loc}{loc}

%Environments
\newcommand{\Theo}[3]{\begin{#1}\label{#2} #3 \end{#1}}
\newcommand{\pf}[1]{\begin{proof} #1 \end{proof}}
\newcommand{\eq}[1]{\begin{equation}\begin{alignedat}{2} #1 \end{alignedat}\end{equation}}
\newcommand{\IntEq}[4]{#1&#2#3	 &\quad &\text{in}~#4,}
\newcommand{\BEq}[4]{#1&#2#3	 &\quad &\text{on}~#4}
\newcommand{\br}[1]{\left(#1\right)}



%Logical symbols
\newcommand{\Ra}{\Rightarrow}
\newcommand{\ra}{\rightarrow}
\newcommand{\hra}{\hookrightarrow}
\newcommand{\mt}{\mapsto}

% Aleksandrov Reflection Macros
\DeclareMathOperator{\reflectionvector}{V}
\DeclareMathOperator{\reflectionangle}{\delta}
\newcommand{\reflectionplane}[1][\reflectionvector]{\ensuremath{P_{#1}}}
\newcommand{\reflectionmap}[1][\reflectionvector]{\ensuremath{R_{#1}}}
\newcommand{\reflectionset}[2][\reflectionvector]{\ensuremath{{#2}_{#1}}}
\newcommand{\reflectionhalfspace}[1][\reflectionvector]{\ensuremath{\reflectionset[{#1}]{H}}}
\DeclareMathOperator{\vertvec}{e}
\DeclareMathOperator{\origin}{O}
\DeclareMathOperator{\radialprojection}{\pi}
\DeclareMathOperator{\height}{h}
\DeclareMathOperator{\equator}{E}
\newcommand{\ip}[2]{\ensuremath{\langle{#1},{#2}\rangle}}
\DeclareMathOperator{\intersect}{\cap}
\DeclareMathOperator{\union}{\cup}
\DeclareMathOperator{\nor}{\nu}
\DeclareMathOperator{\basepoint}{p_0}
\DeclareMathOperator{\radialdistance}{r}

%Fonts
\newcommand{\mc}{\mathcal}
\renewcommand{\it}{\textit}
\newcommand{\mrm}{\mathrm}

%Spacing
\newcommand{\hp}{\hphantom}


\parindent 0 pt

\protected\def\ignorethis#1\endignorethis{}
\let\endignorethis\relax
\def\TOCstop{\addtocontents{toc}{\ignorethis}}
\def\TOCstart{\addtocontents{toc}{\endignorethis}}

%\usepackage[left=1in,right=1in,top=1in,bottom=1in]{geometry}
\begin{document}

\title[]
 {On the classification of ancient solutions to curvature flows on the sphere}

\curraddr{}
\email{}
\date{\today}

\dedicatory{}
\subjclass[2010]{}
\keywords{}

\begin{abstract}
We consider the evolution of hypersurfaces on the unit sphere $\mathbb{S}^{n+1}$ by 1-homogeneous, convex functions of principal curvatures. Our main result states that the only convex, ancient solutions of shrinking curvature flows with 1-homogeneous, convex speeds are shrinking geodesic spheres. The main tools are  differential Harnack inequalities, a rigidity result in the sphere, and an Aleksandrov reflection argument. We also obtain Harnack inequalities for flows by $\a$-power of the mean curvature with $0<\alpha<1$. We introduce the notion of `quasi-ancient` solutions, which play a similar role to ancient solutions for flows that do not admit non-trivial ancient solutions. The techniques presented here in fact allow us to prove that any convex, quasi-ancient solution of a curvature flow which satisfies a uniform bound on the second fundamental form backwards in time must be a family of shrinking geodesic spheres. As an example we treat quasi-ancient solutions of $H^{\a}$-flow with $0<\a<1$.
\end{abstract}

\maketitle

\section{Introduction}

We consider the evolution of a hypersurface $M^n$ by
\eq{\label{eq:CurvFlow}
\partial_tx=-\varphi(f)\nu,~ x:M^n\times[0,T)\to M_K,
}
where \(M_K\) is the simply connected space form of constant sectional curvature \(K\), and $f\in C^{\8}(\G_+)$ is a strictly monotone, 1-homogeneous, symmetric function on the eigenvalues of Weingarten map \(\mathcal{W}\) (principal curvatures) \(\kappa_1, \cdots, \kappa_n\).


\section{Ancient and quasi-ancient Solutions}

\subsection{Backwards Limit}
We consider the spherical ambient space, $K=1,$ without further mention, and we are interested in solutions with maximal possible lifetime. To understand this maximal time, we define $T_S$ to be the lifespan of \it{the} convex spherical solution of \eqref{eq:CurvFlow}. By the convex spherical solution we mean a family of geodesic spheres shrinking under the flow \eqref{eq:CurvFlow} collapsing to a point at time $t=0$ and existing on the maximal interval \((-T_S, 0)\). For 1-homogeneous $\p$, \(T_S = \infty\), but for $\alpha$-homogeneous $\p$ with $\alpha<1$, \(T_S\) is finite.

\begin{lemma}
 Consider \eqref{eq:CurvFlow} with speed \(\p = H^{\alpha}\) for \(\alpha \in (0,1)\). Then a flow of strictly convex geodesic spheres has a finite lifespan, i.e., let $S_r(p)$ be a geodesic sphere in $\S^{n+1}$ around $p\in \S^{n+1}$. Then the flow exists only for a finite time interval \((-T_S,0)\) with \(0 < T_S < \infty\), collapsing to a point at \(t=0\) and converging to an equator at \(t=T_S\).
\end{lemma}

\begin{proof}
Since $H$ is constant on a geodesic sphere, for a spherical flow the evolution equation for $\p=H^{\a}$ yields
\eq{\fr{d}{dt}H^{\a}\geq \a n H^{2\a-1}.}
This yields
\eq{\fr{d}{dt}H\geq nH^{\a}.}
Since the right hand side remains strictly positive under this ODE we obtain finite lifespan forward in time.
Convexity and integration over some interval $(a,b)$ yield
\eq{0\leq H^{1-\a}(a)\leq H^{1-\a}(b)-(1-\a)n(b-a).}
Letting $a\ra-\8$ gives finite existence backwards in time.
\end{proof}

\begin{lemma}
Let $x$ be a convex solution of \eqref{eq:CurvFlow}, defined on the open interval $(-T,0),$ where $0$ is the collapsing time, then $T\leq T_S.$
\end{lemma}

\begin{proof}
Suppose $T>T_S+\e$ for some $\e>0$. Since $M=M_{-T_S-\fr{\e}{2}}$ bounds a convex body $\hat{M}$, it strictly contained in an open hemisphere due to the classical paper \cite{CarmoWarner:/1970}. Then there exists a geodesic sphere $S$ with $\hat{M}\sub\hat{S}.$ By the avoidance principle the flow with initial hypersurface $M$ collapses before the spherical flow contradicting $T>T_S$.
\end{proof}

Due to this lemma the following definition is reasonable.

\begin{definition}
A convex solution of \eqref{eq:CurvFlow} defined on an interval $(-T,0)$ is called \it{quasi-ancient}, if $T=T_S$.
\end{definition}
The term ancient is reserved for the situation when \(T_S=\infty\), and by the definition, ancient solutions are also quasi-ancient.

The aim of this section is to prove that for a quasi-ancient solution of \eqref{eq:CurvFlow} the backwards limit of the flow hypersurfaces \it{with bounded mean curvature}, $M_t$ is an equator for $t\ra -T_S$. We will use the method of \cite{MakowskiScheuer:/2013} to achieve this.
For 1-homogeneous, convex speeds, Proposition \ref{cor:boundedH} gives a bound on mean curvature for ancient solutions. For quasi-ancient solutions, the Harnack inequality does not in general give such a bound, since we cannot send \(t \to -\infty\) whenever \(T_S\) is finite. One can envisage backwards limits as convex polyhedra and hence with unbounded \(H\), but it is not clear that these can arise as backwards limits of quasi-ancient solutions. Thus at this stage we must make the additional assumption that \(H\) is bounded for quasi-ancient solutions.
\begin{proposition}
\label{cor:boundedH}
Suppose $f$ is a convex curvature function. Then any convex ancient solution of the contracting flow with speed $\p(f)=f$ satisfies
\[\partial_t \p-b^{ij}\nabla_i\p\nabla_j\p\geq 0.\]
In particular, for all $t\le -1$ we have
$H(\cdot,t)\leq c.$
Here $c<\infty$ depends only on $M_{-1}.$
\end{proposition}
\begin{proof}
For any $t>s$, the  Harnack estimate of Theorem \ref{thm: main A} implies that
$$\partial_t \p-b^{ij}\nabla_i\p\nabla_j\p+\frac{n}{2}\frac{\p}{t-s}>0.$$
Allowing $s\to-\infty$ proves the first claim. For the second claim, observe that for any 1-homogeneous convex $f$ we have \[f\ge \frac{f(1,\cdots,1)}{n}H,\]
see \cite[Chapter 2]{Gerhardt:/2006}. Therefore, ancient solutions satisfy
\[H(\cdot,t)\leq \frac{n}{f(1,\cdots,1)}\p(\cdot,t)\leq \frac{n}{f(1,\cdots,1)}\p(\cdot,0). \]
\end{proof}

\begin{lemma}\label{ISC}
Let $x$ be a quasi-ancient solution of \eqref{eq:CurvFlow}. Then there holds:\\

(i)~For all $t_0<0$ there exists a uniform radius $R>0,$ such that the enclosed convex bodies $\hat{M}_t,$ $-T_S<t\leq t_0,$ of the flow hypersurfaces $M_t$ satisfy a uniform interior sphere condition with radius $R.$\\

(ii)~For every $y_0\in\mrm{int}~\hat{M}_{t_0}$ the hypersurfaces $M_t,$ $-T_S<t\leq t_0$ can be written as a graph in geodesic polar coordinates around $y_0$ and the corresponding graph functions satisfy uniform $C^2$-estimates.
\end{lemma}

\pf{
Fix an interior point $y_0\in \mrm{int}~\hat{M}_{t_0}.$ Since for a contracting flow the enclosed convex bodies of the flow hypersurfaces are strictly decreasing, they are strictly increasing backwards in time. By \cite[Lemma~3.9]{MakowskiScheuer:/2013} there exists a closed hemisphere $\mc{H}(x_0),$ such that
\eq{\hat{M}_t\sub\mc{H}(x_0).}
In our situation all hypersurfaces $M_t,$ $-T_S<t\leq t_0,$ satisfy
\eq{B_{\e}(y_0)\sub \mrm{int}~\hat{M}_t}
and
\eq{B_{\e}(\hat{y}_0)\sub \hat{M}_t^c}
with a uniform $\e,$ where $\hat{y}_0$ denotes the antipodal point of $y_0.$
Now we prove the two claims.\\

(i)~Consider the stereographic projection with $\hat{y}_0$ corresponding to infinity. The image hypersurfaces are then strictly convex hypersurfaces in Euclidean space with uniformly bounded second fundamental form. Blaschke's rolling theorem, cf.~\cite{Blaschke:/1956}, gives the interior sphere condition.\\

(ii)~
Write the $M_t$ as graphs in geodesic polar coordinates around $y_0,$
\eq{M_t=\{(r,x^i)\cn r=u(t,x^i)\},}
where $r$ describes the geodesic distance to $y_0.$ In these coordinates the spherical metric takes the form
\eq{d\-s^2=dr^2+\sin^{2}r\s_{ij}dx^idx^j,}
where $(\s_{ij})$ is the round metric of $\S^n.$

Hence on the set in which the $M_t$ range, the metrics $\-g_{ij}=\sin^2r\s_{ij}$ and $\s_{ij}$ are equivalent.
Due to \cite[Thm.~2.7.10]{Gerhardt:/2006} for all convex hypersurfaces $M_t$ the quantity
\eq{v^2=1+\-g^{ij}\nabla_iu\nabla_ju}
is uniformly bounded by a constant which only depends on $\e.$
Hence by the equivalence of norms the $M_t$ are uniformly $C^1$-bounded in the sense that the corresponding functions $u(t,\cdot)$ are uniformly $C^1(\S^n)$-bounded.
A straightforward computation yields the following representation of the Weingarten map in terms of the function $u,$ namely
\eq{h^i_j=\fr{\vt'}{v\vt}\d^i_j+\fr{\vt'}{v^3\vt^3}\nabla^iu\nabla_ju-\fr{\~g^{ik}}{v\vt^2}\nabla^2_{kj}u,}
where $\~g^{ij}$ is the inverse of $\~g_{ij}=\vt^{-2}g_{ij},$ $\vt(r)=\sin r$ and covariant derivatives as well as index raising is performed with respect to $\s_{ij},$ compare for example \cite[(3.82)]{Scheuer:05/2015}. Due to the curvature estimates we obtain uniform $C^2(\S^n)$-estimates.
}

\begin{corollary}\label{Backlimit}
Let $x$ be a quasi-ancient solution of \eqref{eq:CurvFlow}. Then there exists a unique backwards limiting hypersurface $M_{-T_S}$ and the flow hypersurfaces $M_t$ converge to $M_{-T_S}$ in $C^{1,\b},$ $0<\b<1,$ in the sense that for a common graph representation as in Lemma \ref{ISC} there holds
\eq{u(t,\cdot)\ra u(-T_S,\cdot)}
in the norm of $C^{1,\b}(\S^n).$
\end{corollary}

\pf{
Due to the pointwise monotonicity of $u(t,\cdot)$ backwards in time we obtain a pointwise limit. The $C^{1,\b}$-convergence follows from compactness.
}

\begin{theorem}
\label{thm:backwardslimit}
The hypersurface $M_{-T_S}$, defined in Corollary \ref{Backlimit}, is an equator.
\end{theorem}

\pf{
Since the convex bodies $\hat{M_t}$ are increasing backwards in time and due to the uniform convergence of $M_t$ to $M_{-T_S},$ the set
\eq{\hat{M}_{-T_S}:=\overline{\bigcup_{t<0}\hat{M}_t}}
is a compact body with
\eq{\del \hat{M}_{-T_S}=M_{-T_S}.}
Since $\mrm{int}(\hat{M}_{-T_S})$ is a strictly convex set, it is especially weakly convex in a hemisphere in the sense of \cite[Def.~3.2]{MakowskiScheuer:/2013}. Thus $\hat{M}_{-T_S}$ is a weakly convex body in a hemisphere. The proof of \cite[Lemma~6.1]{MakowskiScheuer:/2013} can literally be applied to show that $\hat{M}_{-T_S}$ satisfies a uniform interior sphere condition as well.
We can apply \cite[Thm.~1.1]{MakowskiScheuer:/2013} and obtain that $\hat{M}_{-T_S}$ is either strictly contained in an open hemisphere or is equal to a closed hemisphere. The first alternative is not possible since the solution is quasi-ancient. We conclude that $\del \hat{M}_{-T_S}=M_{-T_S}$ is an equator of $\S^{n+1}.$
}


\subsection{Aleksandrov Reflection}

In this section, we use Theorem \ref{thm:backwardslimit} to classify convex, embedded, (quasi-)ancient solutions of contracting curvature flows on \(\S^{n+1}\) as either equators or shrinking geodesic spheres. The proof uses Aleksandrov reflection as in \cite{bryanlouie,2015arXiv150802821B}. Here we give a very general version with minimal assumptions on the flow: all we require is that the flow limits to an equator at $-T_S$ and that the maximum principle holds forward in time.

We begin with some preliminaries of the Aleksandrov reflection on \(\S^{n+1}\). First, we will work relative to the limiting equator obtained in Theorem \ref{thm:backwardslimit}, denoted by $\equator = M_{-T_S}$. The equator \(\equator\) determines two \emph{open} hemispheres \(H^{\pm}\) with centers \(\pm \basepoint\) and we assume the flow is contained in the upper hemisphere, $M_t \subset H^+$. It's convenient to make use of the ambient Euclidean space, \(\R^{n+2}\) with \(\S^{n+1} \subset \R^{n+2}\) via the standard embedding. Let \(\vertvec = \overrightarrow{\origin\basepoint}\) be the unit vector in \(\R^{n+2}\) that points from the origin \(\origin\) to \(\basepoint\), the ``vertical direction''. 

To define the Aleksandrov reflection, let \(\reflectionvector \in \R^{n+2}\) be any unit vector satisfying \(\ip{\reflectionvector}{\vertvec} \leq 0\). Let \(\reflectionplane = \reflectionvector^{\perp}\) be the hyperplane through the origin orthogonal to \(\reflectionvector\). Let \(\reflectionhalfspace^{\pm} = \{\pm \ip{x}{\reflectionvector} > 0\}\) denote the open halfspaces with boundary \(\reflectionplane\). For any subset \(S \subset \S^{n+1}\), write \(\reflectionset{S}^{\pm} = S \intersect \reflectionhalfspace^{\pm}\). Lastly, let \(\reflectionangle \geq 0\) denote the angle \(\reflectionvector\) makes with \(\equator\) so that \(\sin \reflectionangle = \ip{\reflectionvector}{-\vertvec}\).

\begin{definition}
The Aleksandrov reflection across \(\reflectionplane\) is the map defined by
\[
\reflectionmap: x \in \R^{n+2} \mapsto x - 2\ip{x}{\reflectionvector} \reflectionvector.
\]
\end{definition}

This map is an idempotent, (orientation reversing) isometry of \(\R^{n+2}\) fixing \(\reflectionplane\) and in particular fixing the origin. Therefore, it induces an idempotent isometry of \(\S^{n+1}\). Our aim is to show that for any \(\reflectionvector\) with \(\reflectionangle = 0\), the flow \(M_t\) is invariant under \(\reflectionmap\). This will complete the classification since the invariance implies that at each time \(t\), \(M_t\) is contained in a hyperplane orthogonal to \(\vertvec\) and hence must be a geodesic sphere. To achieve this goal, we first work with perturbed reflections coming from the condition \(\reflectionangle > 0\), obtain estimates and then finally send \(\reflectionangle \to 0\). 

Let \(\radialdistance(x) = d_{\S^{n+1}} (\basepoint, x)\) denote the spherical distance from \(\basepoint\) to \(x \in \S^{n+1}\). The radial projection onto \(\equator\) is the map \(x \in \S^{n+1} \mapsto \radialprojection(x) \in \equator\), where \(\radialprojection\) is the nearest point on \(\equator\) to \(x\). If \(x \ne \pm \basepoint\), then \(\radialprojection(x)\) is a single point. If \(x = \pm \basepoint\), then \(\radialprojection(x) = \equator\). In any event, given \(y \in \radialprojection(x)\), there is a unique length minimizing geodesic joining \(x\) to \(y\) and this geodesic must pass through \(\pm \basepoint\). The height function is \(\height(x) = \ip{x}{\vertvec}\) and is related to the radial distance via
\[
\height(x) = \cos(\radialdistance(x))
\]
which is monotonically decreasing in \(\radialdistance\).

For \(x \in \equator\), we have \(\ip{x}{\vertvec} = 0\) and
\[
\height(\reflectionmap(x)) = \ip{\vertvec}{x - 2 \ip{x}{\reflectionvector} \reflectionvector} = 2 \sin\reflectionangle \ip{x}{\reflectionvector}.
\]
In the case \(x \in \reflectionset{\equator}^+\), we have \(\ip{x}{\reflectionvector} > 0\) and hence \(\height(\reflectionmap(x)) > 0\). In the case \(x \in \equator \intersect \reflectionplane\), we have \(\ip{x}{\reflectionvector} = 0\) and hence \(\height(\reflectionmap(x)) = 0\). This allows us to compare \(\reflectionset{(\reflectionmap(\equator))}^+\) where \(\height>0\) with \(\reflectionset{\equator}^-\) where \(\height = 0\). For any hypersurface close to \(\equator\) we will then obtain a similar relation. The major difficulty occurs on the boundary \(\reflectionplane\) where both sets have \(\height = 0\). To make this notion precise, we define the following relation:

\begin{definition}
For subsets \(S,T \subset \S^{n+1}\), we say \emph{\(S\) one-sided reflects above \(T\)}, written \(\reflectionmap(S^+) \geq T^-\) provided \(\radialdistance(x) \leq \radialdistance(y)\) for every \(x \in S^+\) and every \(y \in \radialprojection^{-1} (\reflectionmap(x)) \intersect T^-\). Equivalently we may require \(\height(x) \geq \height(y)\).
\end{definition}

In other words, on the minus side of \(\reflectionplane\), the reflection \(\reflectionmap(S)\) lies ``above'' \(T\). Assume that \(M_t \to \equator\) in \(C^1\) as \(t \to -T_S\) and that \(M_t\) evolves by a parabolic equation, uniform in any region with \(h \geq C g\) for \(C>0\). In the following sequence of lemmas, under these assumptions, we prove that for a fixed \(\reflectionangle \in (0,\pi/4)\) the hypersurface \(M_t\) one-sided reflects above itself on an interval \((-T_S, T)\) for any \(\reflectionangle \in (0, \reflectionangle_0)\) with \(T\) independent of \(\reflectionangle\).

\begin{lemma}
\label{lem:approximate_symmetry}
For any \(\reflectionangle \in (0,\reflectionangle_0)\) there exists a \(T_{\reflectionangle} \in (0, T_S)\) such that \((\reflectionset{\reflectionmap(M_t))}^- \geq M_t^-\) for all \(t \in (-T_S, -T_{\reflectionangle})\).
\end{lemma}

\begin{proof}
Since \(M_t\) converges in \(C^1\) to \(E\) as \(t\to-T_S\), we may write \(M_t\) as the graph of a \(C^1\), positive function over \(\equator\) in geodesic polar coordinates, \(M_t = \{(f_t(\sigma), \sigma) \in (0,\pi) \times \equator\)\} for all \(t \in (-T_S, T_{\reflectionangle})\) with \(T_{\reflectionangle}\) sufficiently close to \(T_S\). Moreover, for \(\reflectionangle \in (0,\reflectionangle_0)\), \(\reflectionmap(\equator) = \{(g_{-T_S}(\sigma), \sigma)\)\} is a graph over \(\equator\) with \(\ip{\nu}{\vertvec}\) uniformly bounded below where \(\nu\) is the unit normal to \(\reflectionmap(\equator)\). Since \(\reflectionmap\) is an isometry, \(\reflectionmap(M_t) \to \reflectionmap(\equator)\) in \(C^1\) as \(t \to -T_S\) and hence, increasing \(T_{\reflectionangle}\) if necessary, we also have that \(\reflectionmap(M_t) = \{(g_t(\sigma), \sigma) \in (0,\pi) \times \equator\)\} is also a graph for all \(t \in (-T_S, T_{\reflectionangle})\).

As noted above, \(\height(\reflectionmap(x)) > 0\) on \(\equator^+\) and \(\height(x) = \cos(\radialdistance(x))\). Thus, continuity implies that for \(\epsilon > 0\) there exits an \(\eta>0\) such that \(\radialdistance(\reflectionmap(x)) < \pi/2 - \epsilon\) provided \(x \in \equator_{\eta} = \{x \in E: d(x, E \intersect \reflectionplane) > \eta\}\). Possibly by making \(T_{\reflectionangle}\) smaller again, since \(M_t \to_{C^1} \equator\), we can arrange that \(d(M_t, \equator) < \epsilon/2\) for all \(t < T_{\reflectionangle}\); that is, \(\radialdistance (x) > \pi/2 - \epsilon/2\) for all \(x \in M_t\). Now for \(x \in M_t^+ \intersect \radialprojection^{-1} \equator_{\eta}\), since \(\reflectionmap\) is an isometry, we have \(d(\reflectionmap(x), \reflectionmap(\radialprojection(x))) < \epsilon/2\); therefore, \(\radialdistance(\reflectionmap(x)) < \pi/2 - \epsilon/2\). Consequently, away from the strip \(\{x \in \equator: d(x, \equator \intersect \reflectionplane) \leq \eta\}\), we have \(\radialdistance(\reflectionmap(\reflectionset{(M_t)}^+)) < \pi/2 - \epsilon/2\) and \(\radialdistance(M_t^-) > \pi/2 - \epsilon/2\). That is, away from the strip, we have \(\reflectionmap(\reflectionset{(M_t)}^+) > \reflectionset{(M_t)}^-\).

Now let \(C\subset \equator\) be any great circle and consider \(f_t|_C\). Using the Backwards Approximate Symmetry Lemma \cite[Lemma 5.1]{bryanlouie} (which applies whenever \(f_t\) converges in \(C^1\) to \(C\)), we find that (possibly by decreasing \(T_{\reflectionangle}\)) again, over \(C\) we have
\[
\reflectionmap(\reflectionset{(M_t)}^+) \geq \reflectionset{(M_t)}^-
\]
on the strip \(\{x \in \equator: d(x, \equator \intersect \reflectionplane) \leq \eta\}\). Here, \(T_{\reflectionangle}\) depends only on \(\|f_t - \pi/2\|_{C^1}\) and \(\reflectionangle\); therefore, \(T_{\reflectionangle} \in (0, T_S)\) is independent of \(C\). Thus we find that
\begin{equation}
\label{eq:backwards_approximate_symmetry}
\reflectionmap(\reflectionset{(M_t)}^+) \geq \reflectionset{(M_t)}^-
\end{equation}
everywhere for all \(t \in (-T_S, -T_{\reflectionangle})\).
\end{proof}

\begin{lemma}
\label{lem:approximate_symmetrypreserved}
There exists a \(T\) independent of \(\reflectionangle\) such that for any \(\reflectionangle \in (0,\reflectionangle_0)\) we have \(\reflectionset(M_t^+) \geq M_t^-\) for all \(t \in (-T_S, -T)\).
\end{lemma}

\begin{proof}
Let us now define \(T_{\reflectionangle}\) so that \((-\infty, T_{\reflectionangle})\) is the largest interval on which the relation \eqref{eq:backwards_approximate_symmetry} holds. Also define \(T = \inf\limits_{\reflectionangle \in (0,\reflectionangle_0)} T_{\reflectionangle}\). We want to show that \(T > -\infty\), and hence that the relation \eqref{eq:backwards_approximate_symmetry} holds on the non-empty, open interval \((-\infty, T)\). To show $T>-\infty$, we apply the maximum principle as in \cite[Lemma 5.2]{bryanlouie}: We know that \(\reflectionset{(M_t)}^-\) and \(\reflectionmap(\reflectionset{(M_t)}^+)\) lie in the interior of \(\reflectionhalfspace^-\) with the common boundary lying in \(\reflectionplane \intersect \S^{n+1}\). We also know that in a neighborhood of \(\reflectionplane \intersect \S^{n+1}\), the hypersurfaces \(\reflectionset{M_t}^-\) and \(\reflectionmap(\reflectionset{(M_t)}^+)\) are disjoint for \(t\) sufficiently negative (depending on \(\reflectionangle\)) because \(\reflectionset{M_t}^-\) is \(C^1\)-close to the equator, while \(\reflectionmap(\reflectionset{(M_t)}^+)\) is \(C^1\)-close to the reflected equator. The strong maximum principle, using a parabolic version of the Hopf boundary point lemma (similar to \cite[Theorem 2.2]{MR1483984}), ensures that the  relation \(\reflectionmap(\reflectionset{(M_t)}^+) \geq \reflectionset{(M_t)}^-\) is preserved along the flow as long as both \(\reflectionset{M_t}^-\) and \(\reflectionmap(\reflectionset{(M_t)}^+)\) are non-empty and intersect \(\reflectionplane\) transversely, and these latter conditions are true on an open interval \((-\infty, S)\) with \(S>-\infty\) independent of \(\reflectionangle\) (see \cite[Lemma 5.2]{bryanlouie}). In summary, since the relation \eqref{eq:backwards_approximate_symmetry} is true on \((-\infty, \min\{T_{\reflectionangle},S\})\), the maximum principle applied in the time interval $[\min\{T_{\reflectionangle},S\}/2,S)$ ensures that
\begin{equation*}
\label{eq:longtime_approximate_symmetry}
\reflectionmap(\reflectionset{(M_t)}^+) \geq \reflectionset{(M_t)}^-
\end{equation*}
for all \(t \in (-\infty, S)\) and any \(\reflectionangle \in (0,\reflectionangle_0)\). Hence $T\ge S.$
\end{proof}

\begin{theorem}
\label{thm:classification}
Let \(M_t\) be a convex, embedded (quasi-)ancient solution of any parabolic equation on \(\S^{n+1}\) such that \(M_t \to \equator\) in \(C^1\) as \(t \to -T_S\). Then \(M_t\) is a family of shrinking geodesic spheres emanating from the equator \(\equator\) at \(t=-T_S\).
\end{theorem}

\begin{proof}
Now we send \(\reflectionangle \to 0\) to complete the proof using \cite[Proposition 5.3]{bryanlouie} (which applies in any dimension) to conclude that \(M_t\) is a geodesic sphere for all \(t \in (-\infty, T)\) and thus for all negative times by the uniqueness of solutions.
\end{proof}


\bibliographystyle{amsplain}
\bibliography{Bibliography.bib}


\end{document}
