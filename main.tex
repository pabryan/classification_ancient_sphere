\documentclass[12pt]{amsart}

%\documentclass[10 pt]{amsart}

\usepackage[ocgcolorlinks,linktoc=all]{hyperref}
\hypersetup{citecolor=blue,linkcolor=red}
\usepackage[parfill]{parskip}
\usepackage{graphicx}

%\usepackage{amsthm}
\usepackage{cleveref}
\crefname{lemma}{Lemma}{Lemmata}
\crefname{equation}{equation}{equations}

\newtheorem{theorem}{Theorem}
\newtheorem{lemma}[theorem]{Lemma}
\newtheorem{proposition}[theorem]{Proposition}
\newtheorem{corollary}[theorem]{Corollary}

\newtheorem*{thmA}{Theorem}
\newtheorem*{thmB}{Theorem}
\newtheorem*{rem}{Remark}
\newtheorem*{thmmain}{Theorem}
\newtheorem*{propmain}{Proposition}

\theoremstyle{definition}
\newtheorem{definition}[theorem]{Definition}
\newtheorem{example}[theorem]{Example}
\newtheorem{xca}[theorem]{Exercise}

\theoremstyle{remark}
\newtheorem{remark}[theorem]{Remark}

\numberwithin{equation}{section}

%Symbols
\renewcommand{\~}{\tilde}
\renewcommand{\-}{\bar}
\newcommand{\bs}{\backslash}
\newcommand{\cn}{\colon}
\newcommand{\sub}{\subset}

\newcommand{\N}{\mathbb{N}}
\newcommand{\R}{\mathbb{R}}
\newcommand{\Z}{\mathbb{Z}}
\renewcommand{\S}{\mathbb{S}}
\renewcommand{\H}{\mathbb{H}}
\newcommand{\C}{\mathbb{C}}
\newcommand{\K}{\mathbb{K}}
\newcommand{\Di}{\mathbb{D}}
\newcommand{\B}{\mathbb{B}}
\newcommand{\8}{\infty}

%Greek letters
\renewcommand{\a}{\alpha}
\renewcommand{\b}{\beta}
\newcommand{\g}{\gamma}
\renewcommand{\d}{\delta}
\newcommand{\e}{\epsilon}
\renewcommand{\k}{\kappa}
\renewcommand{\l}{\lambda}
\renewcommand{\o}{\omega}
\renewcommand{\t}{\theta}
\newcommand{\s}{\sigma}
\newcommand{\p}{\varphi}
\newcommand{\z}{\zeta}
\newcommand{\vt}{\vartheta}
\renewcommand{\O}{\Omega}
\newcommand{\D}{\Delta}
\newcommand{\G}{\Gamma}
\newcommand{\T}{\Theta}
\renewcommand{\L}{\Lambda}

%Mathematical operators
\newcommand{\INT}{\int_{\O}}
\newcommand{\DINT}{\int_{\d\O}}
\newcommand{\Int}{\int_{-\infty}^{\infty}}
\newcommand{\del}{\partial}

\newcommand{\inpr}[2]{\left\langle #1,#2 \right\rangle}
\newcommand{\fr}[2]{\frac{#1}{#2}}
\newcommand{\x}{\times}

\DeclareMathOperator{\dive}{div}
\DeclareMathOperator{\id}{id}
\DeclareMathOperator{\pr}{pr}
\DeclareMathOperator{\Diff}{Diff}
\DeclareMathOperator{\supp}{supp}
\DeclareMathOperator{\graph}{graph}
\DeclareMathOperator{\osc}{osc}
\DeclareMathOperator{\const}{const}
\DeclareMathOperator{\dist}{dist}
\DeclareMathOperator{\loc}{loc}

%Environments
\newcommand{\Theo}[3]{\begin{#1}\label{#2} #3 \end{#1}}
\newcommand{\pf}[1]{\begin{proof} #1 \end{proof}}
\newcommand{\eq}[1]{\begin{equation}\begin{alignedat}{2} #1 \end{alignedat}\end{equation}}
\newcommand{\IntEq}[4]{#1&#2#3	 &\quad &\text{in}~#4,}
\newcommand{\BEq}[4]{#1&#2#3	 &\quad &\text{on}~#4}
\newcommand{\br}[1]{\left(#1\right)}



%Logical symbols
\newcommand{\Ra}{\Rightarrow}
\newcommand{\ra}{\rightarrow}
\newcommand{\hra}{\hookrightarrow}
\newcommand{\mt}{\mapsto}

% Aleksandrov Reflection Macros
\DeclareMathOperator{\reflectionvector}{V}
\DeclareMathOperator{\reflectionangle}{\delta}
\newcommand{\reflectionplane}[1][\reflectionvector]{\ensuremath{P_{#1}}}
\newcommand{\reflectionmap}[1][\reflectionvector]{\ensuremath{R_{#1}}}
\newcommand{\reflectionset}[2][\reflectionvector]{\ensuremath{{#2}_{#1}}}
\newcommand{\reflectionhalfspace}[1][\reflectionvector]{\ensuremath{\reflectionset[{#1}]{H}}}
\DeclareMathOperator{\vertvec}{e}
\DeclareMathOperator{\origin}{O}
\DeclareMathOperator{\radialprojection}{\pi}
\DeclareMathOperator{\height}{h}
\DeclareMathOperator{\equator}{E}
\newcommand{\ip}[2]{\ensuremath{\langle{#1},{#2}\rangle}}
\DeclareMathOperator{\intersect}{\cap}
\DeclareMathOperator{\union}{\cup}
\DeclareMathOperator{\nor}{\nu}
\DeclareMathOperator{\basepoint}{p_0}
\DeclareMathOperator{\radialdistance}{r}

%Fonts
\newcommand{\mc}{\mathcal}
\renewcommand{\it}{\textit}
\newcommand{\mrm}{\mathrm}

%Spacing
\newcommand{\hp}{\hphantom}


\parindent 0 pt

\protected\def\ignorethis#1\endignorethis{}
\let\endignorethis\relax
\def\TOCstop{\addtocontents{toc}{\ignorethis}}
\def\TOCstart{\addtocontents{toc}{\endignorethis}}

\usepackage[left=1in,right=1in,top=1in,bottom=1in]{geometry}
\begin{document}

\title[]
 {On the classification of ancient solutions to curvature flows on the sphere}

\curraddr{}
\email{}
\date{\today}

\dedicatory{}
\subjclass[2010]{}
\keywords{}

\begin{abstract}
We consider the evolution of hypersurfaces on the unit sphere $\mathbb{S}^{n+1}$ by 1-homogeneous, convex functions of principal curvatures. Our main result states that the only convex, ancient solutions of shrinking curvature flows with 1-homogeneous, convex speeds are shrinking geodesic spheres. The main tools are  differential Harnack inequalities, a rigidity result in the sphere, and an Aleksandrov reflection argument. We also obtain Harnack inequalities for flows by $\a$-power of the mean curvature with $0<\alpha<1$. The techniques presented here in fact allow us to prove any convex (quasi) ancient solution of a curvature flow which satisfies a uniform bound on the second fundamental form backwards in time must be shrinking geodesic spheres, as an example we treat (quasi) ancient solutions of $H^{\a}$-flow with $0<\a<1$. 
\end{abstract}

\maketitle

\section{Introduction}

We consider the evolution of a hypersurface $M^n$ by
\eq{\label{eq:CurvFlow}
\partial_tx=-\varphi(f)\nu,~ x:M^n\times[0,T)\to M_K,
}
where \(M_K\) is the simply connected space form of constant sectional curvature \(K\), and $f\in C^{\8}(\G_+)$ is a strictly monotone, 1-homogeneous, symmetric function on the eigenvalues of Weingarten map \(\mathcal{W}\) (principal curvatures) \(\kappa_1, \cdots, \kappa_n\).
 Our principal result is Harnack inequalities for flows on the sphere when the speed $\p=f$ is either convex or $\p=H^{\alpha},~ \a\in (0,1)$. This result extends the Harnack inequalities obtained in \cite{2015arXiv150802821B, bryanlouie} on the sphere. Along the way, we obtain new and old Harnack inequalities in Euclidean space \cite{MR1296393, MR1100812, MR1316556, MR2813400, MR1480081}.

\section{Preliminaries}

Let $\overline{g}$ and $\overline{Rm}$ denote, respectively, the metric and the curvature tensor of $M_K$. Define \(M_t := x(M^n,t)\), and let \(g = x_t^{\ast} \overline{g}\) denote the induced metric on \(M\) with $\nabla$ the corresponding Levi-Civita connection. Write $\nu$ for the outer unit normal to $M_t$ and let \(\{\partial_i\}_{i=1}^n\) be a local frame on \(M\) which extends to a frame \(\{\partial_0 = \nu, (x_t)_{\ast} \partial_1, \cdots, (x_t)_{\ast} \partial_n\}\) on \(M_K\) in a neighborhood of \(M_t\).

Let Greek indices range from \(0\) to \(n\) and Latin indices range from \(1\) to \(n\). The Riemann curvature tensor of \(M_K\) satisfies \(\bar{R}_{\alpha\beta\gamma\theta} = K(\bar{g}_{\alpha\gamma}\bar{g}_{\beta\delta} - \bar{g}_{\alpha\theta}\bar{g}_{\beta\gamma})\). We may write the metric $g = \{g_{ij}\}$, second fundamental form $A = \{h_{ij}\}$, the Weingarten map $\mathcal{W} = h^i_j = g^{mi} h_{jm}$ and the Riemann curvature tensor $Rm_{ijkl}$ with respect to the given frame.

We shall write \(\nabla_i\) for covariant derivatives and also use the notation \(\nabla^i = g^{ik} \nabla_k\). Second covariant derivatives will be written \(\nabla^2_{ij} = \nabla_i \nabla_j - \nabla_{\nabla_i \partial j}\) and \((\nabla^2)^i_j = g^{ik} \nabla^2_{kj}\).

The mean curvature of $M^n$ is the trace of the Weingarten map (equivalently the trace of the second fundamental form with respect to $g$), $H = g^{ij}h_{ij} = h^i_i$. We also use the following standard notation
\[
(h^2)_i^j = g^{mj}g^{rs}h_{ir}h_{sm},
\]
\[
(h^2)_{ij} = g_{kj} (h^2)_i^k = h^k_i h_{kj},
\]
\[
|A|^2 = g^{ij}g^{kl}h_{ik}h_{lj} = h_{ij}h^{ij},
\]
Here, $\{g^{ij}\}$ is the inverse matrix of $\{g_{ij}\}.$ For a strictly convex hypersurface, \(A\) is strictly positive-definite and hence has a strictly positive-definite inverse, which we denote by
\[
b = \{b^{ij}\}.
\]

The relations between $A$, $Rm$, and $\overline{Rm}$ are given by the Gau{\ss} and Codazzi equations:
\[
\begin{split}
Rm_{ijkl} &= \overline{Rm}_{ijkl} + h_{ik}h_{jl} - h_{il}h_{jk} \\
&= K(\bar{g}_{ik}\bar{g}_{jl} - \bar{g}_{il}\bar{g}_{jk}) + h_{ik}h_{jl} - h_{il}h_{jk},
\end{split}
\]
\[
\nabla_i h_{jk} = \nabla_k h_{ij},
\]
valid for space forms. We also make use of the Ricci identity,
\[
{Rm^m}_{kij}  = \left(\nabla^2_{i, j} \partial_k - \nabla^2_{j,i} \partial_k\right)^m
\]

We will need some notation for derivatives of the speed \(\varphi\). Let us write
\[
\varphi^{i}_{j} = \frac{\partial \varphi}{\partial h^{j}_{i}}
\]
for the first partial derivatives of \(\varphi\). We may also think of \(\varphi\) as a function of the metric and second fundamental form
\[
\varphi(g, h) = \varphi(g^{ik} h_{kj}).
\]
From this point of view, for the first and second partial derivatives, let us write
\[
\varphi^{ij} = \frac{\partial\varphi}{\partial h_{ij}}, \quad \varphi^{ij,kl} = \fr{\partial^2\varphi}{\partial h_{kl} \partial h_{ij}}.
\]
The trace of \(\varphi^{ij}\) with respect to the metric will be written
\[
\Phi = g_{ij} \varphi^{ij}.
\]
Let us also define the operator
\[
\Box = \varphi^{ij} \nabla^2_{ij}
\]
This operator satisfies the product rule
\begin{equation}
\label{eq:productbox}
\Box (fg) = f \Box g + g \Box f + 2 \varphi^{ij} \nabla_i f \nabla_j g.
\end{equation}

We frequently make use, without comment, of the formula for differentiating an inverse
\[
\frac{\partial g^{ij}}{\partial g_{kl}} = - g^{kj} g^{il}.
\]

First derivatives of \(\varphi\) from the two perspectives are related by
\begin{equation}
\label{eq:delh}
\varphi^{ij} = \frac{\partial \varphi}{\partial h_l^k} \frac{\partial h_l^k}{\partial h_{ij}} = \varphi^l_k g^{ik} \delta^j_l = g^{ik} \varphi^j_k.
\end{equation}
and
\begin{equation}
\label{eq:delg}
\frac{\partial\varphi}{\partial g_{ij}} = \varphi^{l}_{k} \frac{\partial h^{k}_{l}}{\partial g_{ij}} = -\varphi^{l}_{k} g^{ki} g^{rj} h_{rl} = -\varphi^{li}h^{j}_{l}.
\end{equation}
We will also need the mixed second derivatives,
\begin{equation}
\label{eq:delhdelg}
\begin{split}
\frac{\partial \varphi^{ij}}{\partial g_{kl}} &= \frac{\partial}{\partial g_{kl}} \left(g^{sj} \varphi^{i}_{s} \right) = - g^{kj}g^{sl} \varphi^{i}_{s} - g^{sj} (\varphi^i_s)^{mk} h^l_m \\
&= - g^{kj} \varphi^{il} - g^{sj} g_{ns} \varphi^{in,mk} h^l_m \\
&= - g^{kj} \varphi^{il} - \varphi^{ij,mk} h^l_m,
\end{split}
\end{equation}
where we applied \cref{eq:delg} to \(\varphi^i_s\) in the first line.

Covariant derivatives of \(\varphi\) satisfy
\begin{equation}
\label{eq:delphi}
\nabla_k \varphi = \varphi^{ij} \nabla_k h_{ij}
\end{equation}
and the covariant derivative of the trace,
\begin{equation}
\label{eq:delPhi}
\nabla_k \Phi = g_{ij} \varphi^{ij,rs} \nabla_k h_{rs}.
\end{equation}

\section{Basic Evolution equations}

Following \cite{MR1296393, MR1100812, MR1316556, MR1480081}, in this section, we collect basic evolution equations that are needed to calculate the evolution of the following quantities
\[
\chi_1 =t(\partial_t \varphi- b^{ij} \nabla_i \varphi \nabla_j \varphi) +\delta\varphi
\]
and
\[
\chi_2 =t(\partial_t \varphi - b^{ij} \nabla_i \varphi \nabla_j \varphi - K \varphi \Phi) +\delta\varphi
\]
where \(\delta \ne 0\) is an arbitrary, non-zero constant. The evolution equation of $\chi_1$ will be used for obtaining Harnack estimates for convex 1-homogeneous curvature speeds $\p$, and the evolution equation of $\chi_2$ will be used for obtaining stronger Harnack estimates for flow by powers of the mean curvature $\p=H^{\alpha}$ with $\alpha\in(0,1).$ Note that in Euclidean space $\chi_1=\chi_2.$

Let us make a few definitions to keep the calculations more manageable. Let
\[
\alpha_{ij} = \nabla^2_{ij} \varphi + \varphi(h^2)_{ij}, \quad \gamma_{ij} = b^{kl} \nabla_k \varphi \nabla_l h_{ij}, \quad \eta_{ij} = \alpha_{ij} - \gamma_{ij}
\]
and define
\[
\beta = \varphi^{ij} \alpha_{ij} = \Box\varphi + \varphi \varphi^{ij}(h^2)_{ij}, \quad \theta =  b^{ij} \nabla_i \varphi \nabla_j \varphi
\]
so that from the evolution of \(\varphi\) below (\cref{lem:evolution}, \cref{eq:delt_speed}) we may write our main Harnack expressions as
\[
\chi_1 = t(\partial_t\p- \theta) + \delta\varphi
\]
and
\[
\chi_2 = t(\beta - \theta) + \delta\varphi.
\]
We begin by recalling some standard evolution equations and commutators and then break the remaining calculation into several lemmas.

The evolution equations in the following lemma are standard and can be found in many places \cite{MR1296393, MR1100812, MR1316556, MR892052, MR1480081}. The necessary tools are commuting derivatives, using the definition of the curvature tensor for space forms, the Gauss equation, and the Codazzi equation as described in the previous section. Compare also \cite[p.~94-95]{Gerhardt:/2006} and the formula \cite[eq.~(6.17)]{Gerhardt:01/1996}.

\begin{lemma}
\label{lem:evolution}
The following evolution equations hold
\begin{enumerate}
\item \label{eq:delt_metric} $\partial_tg_{ij} = -2\varphi h_{ij}$
\item \label{eq:delt_inversemetric} $\partial_t g^{ij} = 2\varphi h^{ij}$
\item \label{eq:delt_christoffel} $\partial_t {\G}^{k}_{ij} = -\varphi g^{kl} \nabla_l h_{ij} - h^k_i \nabla_j \varphi - h^k_j \nabla_i \varphi + g^{kl} h_{ij} \nabla_l \varphi$
\item \label{eq:delt_sff} $\partial_t h_{ij} = \nabla^2_{ji} \varphi - \varphi(h^2)_{ij} + K \varphi g_{ij}$
\item \label{eq:delt_weingarten} $\partial_t h_i^j = (\nabla^2)^j_i\varphi + \varphi(h^2)_i^j + K \varphi\delta_i^j = \alpha^j_i + K \varphi\delta_i^j$
\item \label{eq:delt_sff_box} \begin{align*}
\partial_t h_{ij} &= \Box h_{ij} + \varphi^{kl} (h^2)_{kl} h_{ij} - (\varphi^{kl}h_{kl} + \varphi) (h^2)_{ij} \\
& \quad + \varphi^{kl,rs}\nabla_i h_{kl}\nabla_j h_{rs} \\
& \quad + K \{(\varphi + \varphi^{kl}h_{kl}) g_{ij} - \Phi h_{ij}\}
\end{align*}
\item \label{eq:delt_weingarten_box} \begin{align*}
\partial_t h_i^j &= \Box h_i^j + \varphi^{kl} (h^2)_{kl} h_i^j - (\varphi^{kl}h_{kl} - \varphi) (h^2)_i^j \\
& \quad + \varphi^{kl,rs}\nabla_i h_{kl}\nabla^j h_{rs} \\
& \quad + K \{(\varphi + \varphi^{kl}h_{kl}) \delta_i^j - \Phi h_i^j\}
\end{align*}
\item \label{eq:delt_inversesff} \begin{align*}
\partial_t b^{ij} &= \Box b^{ij} - \varphi^{rs} (h^2)_{rs} b^{ij} + (\varphi^{kl}h_{kl} + \varphi) g^{ij} \\
& \quad - \left(2b^{lq}\varphi^{kp} + \varphi^{kl,pq}\right) b^{ir}b^{js} \nabla_r h_{kl} \nabla_s h_{pq} \\
& \quad - K \{(\varphi + \varphi^{kl}h_{kl}) b^{ir}b^{j}_{r} - \Phi b^{ij}\}
\end{align*}
\item \label{eq:delt_squaredsff} $\partial_t (h^2)_{ij} = h^k_j \nabla^2_{i,k} \varphi + h^k_i \nabla^2_{j,k} \varphi + h^k_j \varphi(h^2)_{ik} - h^k_i \varphi(h^2)_{jk} + 2K\varphi h_{ij}$
\item \label{eq:delt_speed} $\partial_t \varphi = \Box \varphi + \varphi\varphi^{ij}(h^2)_{ij} + K \varphi\varphi^{ij}g_{ij} = \beta + K\varphi\Phi$
\end{enumerate}
\end{lemma}

We require the commutators \([\nabla, \Box]\) and \([\partial_t, \Box]\). Without further comment we will also use the fact that \([\partial_t, \nabla] f = 0\) for any smooth function \(f\).

\begin{lemma}
\label{lem:gradBox}
For every smooth function $f$, the commutation relation
\[
\begin{split}
([\nabla, \Box]f)_i &= \nabla_i \Box f - (\Box \nabla f)_i = \varphi^{kl,rs} \nabla_i h_{rs} \nabla^2_{kl} f \\
&\quad + K \varphi^{kl} g_{ki} \nabla_l f - K \Phi \nabla_i f \\
&\quad + \varphi^{kl}\left(h^{m}_{l}h_{ki} - h_{kl}h^{m}_{i}\right) \nabla_m f
\end{split}
\]
holds, where \(\nabla f = df\) is the covariant derivative of \(f\) and the subscript \(i\) refers to the \(i\)'th component of a one-form.
\end{lemma}

\begin{proof}
From the Ricci identities
\[
{Rm^m}_{kij}  = \left(\nabla^2_{i, j} \partial_k - \nabla^2_{j,i} \partial_k\right)^m
\]
we obtain that the $3$-tensor $\nabla^3_{kli}f-\nabla^3_{kil}f$
is given by
\[
\nabla^3_{kli}f-\nabla^3_{kil}f={Rm^m}_{kli}\nabla_m f
\]
and thus we obtain
\[
\nabla_i (\varphi^{kl} \nabla^2_{kl} f) - (\varphi^{kl}(\nabla^2_{kl} \nabla f))_i = \varphi^{kl,rs} \nabla_i h_{rs} \nabla^2_{kl}f + \varphi^{kl}{Rm^{m}}_{kli} \nabla_m f.
\]
From the Gauss equation we obtain
\[
\begin{split}
{Rm^{m}}_{kli} \nabla_m f &= \left(K\left(g^{pm}g_{ki}g_{pl}  - g^{pm}g_{pi}g_{kl}\right) + g^{pm} h_{pl}h_{ki} - g^{pm}h_{kl}h_{pi}\right) \nabla_m f \\
&= K\left(g_{ki} \nabla_l f - g_{kl} \nabla_i f\right) + \left(h^{m}_{l}h_{ki} - h_{kl}h^{m}_{i}\right) \nabla_m f.
\end{split}
\]
\end{proof}

\begin{lemma}
\label{lem:deltBox}
The following commutation relation holds
\[
\begin{split}
[\partial_t, \Box] \varphi &= (\partial_{t}\Box - \Box\partial_{t}) \varphi = \varphi^{ij,kl} \nabla^2_{i,j} \varphi (\alpha_{kl} + K \varphi g_{kl}) \\
&\quad + 2\varphi^{ij}h^{k}_{i} (\varphi \nabla^2_{k,j} \varphi + \nabla_k \varphi \nabla_j \varphi) + (\varphi - \varphi^{ij}h_{ij})| \nabla\varphi|^{2}.
\end{split}
\]
\end{lemma}

\begin{proof}
First, let us calculate the evolution of \(\varphi^{ij}\), which will also prove useful later. From the mixed derivative \cref{eq:delhdelg}, the evolution of the metric (\cref{lem:evolution}, \cref{eq:delt_metric}), and the evolution of the second fundamental form (\cref{lem:evolution}, \cref{eq:delt_sff}) we compute
\begin{equation}
\label{eq:deltBox}
\begin{split}
\partial_{t} \varphi^{ij} &= \varphi^{ij,kl} \partial_t h_{kl} + \frac{\partial\varphi^{ij}}{\partial g_{kl}} \partial_t g_{kl} \\
&= \varphi^{ij,kl} \left(\nabla^2_{kl} \varphi - \varphi(h^2)_{kl} + K \varphi g_{kl}\right) + 2\varphi \varphi^{ij,kl} h_{lm}h^{m}_{k} + 2\varphi\varphi^{jk}g^{li}h_{kl} \\
&= \varphi^{ij,kl} \left(\nabla^2_{kl} \varphi + \varphi(h^2)_{kl} + K \varphi g_{kl}\right) + 2\varphi\varphi^{jk}h^{i}_{k} \\
&= \varphi^{ij,kl} \left(\alpha_{kl} + K \varphi g_{kl}\right) + 2\varphi\varphi^{jk}h^{i}_{k}.
\end{split}
\end{equation}

Next, the commutator of \(\partial_t\) and \(\nabla^2_{i,j}\) is given by,
\begin{equation}
\label{eq:deltnabla2}
\begin{split}
\left(\partial_{t}\nabla^2_{i,j} - \nabla^2_{i,j}\partial_{t}\right) \varphi &= \partial_t \left(\nabla_i \nabla_j \varphi - \nabla_{\nabla_i \partial_j} \varphi\right) - \nabla_i \nabla_j \partial_t \varphi + \nabla_{\nabla_i \partial_j} \partial_t \varphi \\
&= - \partial_t \left(\G_{ij}^k \nabla_{\partial_k} \varphi\right) + \G_{ij}^k \nabla_{\partial_k} \partial_t \varphi \\
&= - \nabla_k \varphi \partial_t \G_{ij}^k.
\end{split}
\end{equation}

We obtain from \cref{eq:deltBox}, \cref{eq:deltnabla2}, and the evolution of the Christoffel symbols (\cref{lem:evolution}, \cref{eq:delt_christoffel}),
\[
\begin{split}
\left(\partial_{t}\Box - \Box\partial_{t}\right) \varphi &= \left(\partial_{t}\varphi^{ij}\right) \nabla^2_{i,j} \varphi - \varphi^{ij}\nabla_k \varphi\partial_t \G^{k}_{ij} \\
&= \left[\varphi^{ij,kl} \left(\alpha_{kl} + K \varphi g_{kl}\right) + 2\varphi\varphi^{jk}h^{i}_{k}\right] \nabla^2_{i,j} \varphi \\
&\quad + \varphi^{ij} \nabla_k \varphi \left(\varphi g^{kl} \nabla_l h_{ij} + h^k_i \nabla_j \varphi + h^k_j \nabla_i \varphi - g^{kl} h_{ij} \nabla_l \varphi\right) \\
&= \varphi^{ij,kl} \nabla^2_{i,j} \varphi \left(\alpha_{kl} + K \varphi g_{kl}\right) \\
&\quad + 2\varphi^{jk}h^{i}_{k}\varphi \nabla^2_{i,j} \varphi + h^k_i \varphi^{ij} \nabla_k \varphi \nabla_j \varphi + h^k_j \varphi^{ij} \nabla_k \varphi \nabla_i \varphi \\
&\quad + \varphi g^{kl}\nabla_k \varphi \varphi^{ij} \nabla_l h_{ij} - \varphi^{ij} h_{ij} g^{kl}\nabla_k \varphi \nabla_l \varphi.
\end{split}
\]
The result now follows from \(|\nabla \varphi|^2 = g^{kl}\nabla_k \varphi \nabla_l \varphi\) and \(\nabla_l \varphi = \varphi^{ij} \nabla_l h_{ij}\) by the chain rule.
\end{proof}

The next ingredient is the evolution of the covariant derivative, \(\nabla \varphi = d\varphi\).

\begin{lemma}
\label{lem:Evgradphi}
There holds
\[
\begin{split}
\left((\partial_{t}-\Box)\nabla\varphi\right)_{i} &= \varphi^{kl,rs}\nabla_i h_{rs} \alpha_{kl} + 2 \varphi^{kl} b^{rs} \varphi(h^2)_{rl} \nabla_i h_{ks} \\
&\quad + \varphi^{kl}(h^2)_{kl}\nabla_i \varphi + \left(\varphi^{kl}h^{m}_{l}h_{ki} - \varphi^{kl}h_{kl}h^{m}_{i}\right) \nabla_m \varphi\\
&\quad + K\left(\varphi^{kl}g_{ki} \nabla_l \varphi + \varphi \nabla_i \Phi\right).
\end{split}
\]
\end{lemma}

\begin{proof}
Using the evolution of \(\varphi\) \cref{lem:evolution}, \cref{eq:delt_speed}, the derivative of \(\Phi\) \cref{eq:delPhi}, and the commutator \([\nabla, \Box]\) from \cref{lem:gradBox}, we compute
\[
\begin{split}
\partial_{t}\nabla_i \varphi - (\Box\nabla \varphi)_{i} &= \nabla_i \partial_t \varphi - \nabla_i \Box \varphi + ([\nabla, \Box] \varphi)_i \\
&= \nabla_i \left(\Box\varphi + \varphi^{kl}(h^2)_{kl}\varphi + K \Phi\varphi\right) - \nabla_i (\Box\varphi) \\
&\quad + \varphi^{kl,rs} \nabla_i h_{rs} \nabla^2_{kl} \varphi + K\varphi^{kl}g_{ki} \nabla_l \varphi - K\Phi\nabla_i \varphi \\
&\quad + (\varphi^{kl}h^{m}_{l}h_{ki} - \varphi^{kl}h_{kl}h^{m}_{i}) \nabla_m \varphi \\
&= \varphi^{kl}(h^2)_{kl}\nabla_i \varphi + \varphi^{kl,rs}\nabla_i h_{rs} (h^2)_{kl}\varphi + \varphi\varphi^{kl}(h^s_l \nabla_i h_{ks} + h^r_k \nabla_i h_{rl}) + K \varphi\nabla_i\Phi \\
&\quad + \varphi^{kl,rs} \nabla_i h_{rs} \nabla^2_{kl} \varphi + K\varphi^{kl}g_{ki}\nabla_l \varphi \\
&\quad + (\varphi^{kl}h^{m}_{l}h_{ki} - \varphi^{kl}h_{kl}h^{m}_{i}) \nabla_m \varphi \\
&= \varphi^{kl,rs}\nabla_i h_{rs} \left((h^2)_{kl}\varphi + \nabla^2_{kl} \varphi\right) + 2 \varphi\varphi^{kl} h^s_l \nabla_i h_{ks} \\
&\quad + \varphi^{kl}(h^2)_{kl}\nabla_i \varphi + (\varphi^{kl}h^{m}_{l}h_{ki} - \varphi^{kl}h_{kl}h^{m}_{i}) \nabla_m \varphi \\
&\quad + K \left(\varphi\nabla_i\Phi + \varphi^{kl}g_{ki}\nabla_l \varphi\right),
\end{split}
\]
where in the third equality, we used
\[
\nabla_i (h^2)_{kl} = \nabla_i (g^{sr} h_{ks} h_{rl}) = h^s_l \nabla_i h_{ks} + h^r_k \nabla_i h_{rl}
\]
and in the last equality we used
\[
b^{rs} (h^2)_{rl} = b^{rs} h_{rm} h^m_l = \delta^s_m h^m_l = h^s_l.
\]
\end{proof}
Now we may proceed to the calculations of \(\partial_t \beta\) and \(\partial_t \theta\).
\begin{lemma}
\label{lem:evbeta}
The quantity
\[
\beta = \partial_t \varphi - K\Phi\varphi = \Box\varphi +  \varphi\varphi^{ij} (h^2)_{ij}
\]
satisfies
\[
\begin{split}
(\partial_{t} - \Box)\beta &= \left(\varphi^{ij}(h^2)_{ij} + K\Phi \right)\beta \\
&\quad + (\varphi - \varphi^{ij}h_{ij}) |\nabla\varphi|^{2} + 2\varphi^{ij}h^{k}_{i}\nabla_k \varphi \nabla_j \varphi \\
&\quad + \varphi^{ij,kl} \alpha_{ij} \alpha_{kl} \\
&\quad + 2b^{il}\varphi^{jk} (2\nabla^2_{ij}\varphi\varphi(h^2)_{kl} + \varphi(h^2)_{ij}\varphi(h^2)_{kl}) \\
&\quad + KR_{\beta},
\end{split}
\]
where
\[
R_{\beta} = \varphi \Box \Phi + 2\varphi^{kl} \nabla_k \Phi \nabla_l \varphi + \varphi \varphi^{ij,kl}g_{kl} \alpha_{ij} + 2\varphi^{2}\varphi^{ij}h_{ij}.
\]
\end{lemma}

\begin{proof}
Let us break up the calculation of
\[
(\partial_{t} - \Box)\beta =  \partial_{t}\Box\varphi + \partial_{t} (\varphi\varphi^{ij} (h^2)_{ij}) - \Box\Box\varphi - \Box(\varphi\varphi^{ij} (h^2)_{ij})
\]
into smaller pieces. First, we have lots of nice cancellation. Using the evolution of \(\varphi\) from \cref{lem:evolution}, \cref{eq:delt_speed} and the commutator relation from \cref{lem:deltBox} we have,
\begin{equation}
\label{eq:deltbeta1}
\begin{split}
\partial_{t}\Box\varphi - \Box\Box\varphi - \Box(\varphi\varphi^{ij} (h^2)_{ij}) &= \Box\partial_t\varphi - \Box\Box\varphi - \Box(\varphi\varphi^{ij} (h^2)_{ij}) + [\partial_t, \Box] \varphi \\
&= \Box(\Box \varphi + \varphi\varphi^{ij}(h^2)_{ij} + K \varphi\Phi) - \Box\Box\varphi - \Box(\varphi\varphi^{ij} (h^2)_{ij}) \\
&\quad + \varphi^{ij,kl} \nabla^2_{i,j} \varphi (\alpha_{kl} + K \varphi g_{kl}) \\
&\quad + 2\varphi^{ij}h^{k}_{i} (\varphi \nabla^2_{k,j} \varphi + \nabla_k \varphi \nabla_j \varphi) + (\varphi - \varphi^{ij}h_{ij})| \nabla\varphi|^{2} \\
&= K (\Phi \Box \varphi + \varphi \Box \Phi + 2 \varphi^{kl} \nabla_k \Phi \nabla_l \varphi) \\
&\quad + \varphi^{ij,kl} \nabla^2_{ij} \varphi (\alpha_{kl} + K \varphi g_{kl}) \\
&\quad + 2\varphi^{ij}b^{kl} \varphi (h^2)_{il} \nabla^2_{kj} \varphi + 2\varphi^{ij}h^{k}_{i} \nabla_k \varphi \nabla_j \varphi + (\varphi - \varphi^{ij}h_{ij})| \nabla\varphi|^{2} \\
&= K \Phi \Box \varphi  \\
&\quad + (\varphi - \varphi^{ij}h_{ij})| \nabla\varphi|^{2} \\
&\quad + \varphi^{ij,kl} \nabla^2_{ij} \varphi \alpha_{kl} \\
&\quad + 2\varphi^{ij}b^{kl} \varphi (h^2)_{il} \nabla^2_{kj} \varphi + 2\varphi^{ij}h^{k}_{i} \nabla_k \varphi \nabla_j \varphi \\
&\quad + K(\varphi \Box \Phi + 2 \varphi^{kl} \nabla_k \Phi \nabla_l \varphi + \varphi \varphi^{ij,kl} g_{kl} \nabla^2_{ij} \varphi)
\end{split}
\end{equation}
using, in the third equality, the product rule for \(\Box\) \cref{eq:productbox} and \(h^k_i = h^m_i b^{kl}h_{ml} = b^{kl} (h^2)_{il}\) since \(b\) is the inverse of \(A\).

Next from \cref{eq:deltBox} and \cref{lem:evolution}, \cref{eq:delt_squaredsff} we obtain
\[
\begin{split}
\partial_{t} (\varphi^{ij}(h^2)_{ij}) &= \partial_{t}(\varphi^{ij}) (h^2)_{ij} + \varphi^{ij} \partial_t (h^2)_{ij} \\
&= \left(\varphi^{ij,kl} \left(\alpha_{kl} + K \varphi g_{kl}\right) + 2\varphi\varphi^{jk}h^{i}_{k}\right) (h^2)_{ij} \\
&\quad + \varphi^{ij} \left(h^k_j \nabla^2_{i,k} \varphi + h^k_i \nabla^2_{j,k} \varphi + h^k_j \varphi(h^2)_{ik} - h^k_i \varphi(h^2)_{jk} + 2K\varphi h_{ij}\right) \\
&= \varphi^{ij,kl} (h^2)_{ij} \left(\alpha_{kl} + K \varphi g_{kl}\right) + 2\varphi\varphi^{jk} b^{il} (h^2)_{lk} (h^2)_{ij} \\
&\quad + 2 \varphi^{ij} b^{kl} (h^2)_{il} \nabla^2_{j,k} \varphi  + 2K\varphi\varphi^{ij}h_{ij}
\end{split}
\]
again using \(h^k_i = b^{kl} (h^2)_{il}\) in the last equality.

The remaining term we need to compute is thus
\begin{equation}
\label{eq:deltbeta2}
\begin{split}
\partial_{t} (\varphi \varphi^{ij}(h^2)_{ij}) &= (\partial_{t} \varphi) \varphi^{ij}(h^2)_{ij} + \varphi \partial_t (\varphi^{ij} (h^2)^{ij}) \\
&= (\beta + K \varphi\Phi) \varphi^{ij}(h^2)_{ij} \\
&\quad + \varphi \left[\varphi^{ij,kl} (h^2)_{ij} \left(\alpha_{kl} + K \varphi g_{kl}\right) + 2\varphi\varphi^{jk} b^{il} (h^2)_{lk} (h^2)_{ij} \right.\\
&\quad \left. + 2 \varphi^{ij} b^{kl} (h^2)_{il} \nabla^2_{jk} \varphi  + 2K\varphi\varphi^{ij}h_{ij}\right]. \\
&= (\beta + K \varphi\Phi) \varphi^{ij}(h^2)_{ij} \\
&\quad + \varphi^{ij,kl} \varphi (h^2)_{ij} \alpha_{kl}  \\
&\quad + 2 \varphi\varphi^{ij} b^{kl} (h^2)_{il} \nabla^2_{jk} \varphi + 2\varphi^{jk} b^{il} \varphi (h^2)_{lk} \varphi (h^2)_{ij} \\
&\quad + K \left[\varphi \varphi^{ij,kl} g_{kl} \varphi (h^2)_{ij} + 2K\varphi^2\varphi^{ij}h_{ij}\right].
\end{split}
\end{equation}

Now we add \cref{eq:deltbeta1} and \cref{eq:deltbeta2} together line by line to complete the proof.
\end{proof}

\begin{lemma}
\label{lem:Evtheta}
The quantity $\theta = b^{ij}\nabla_i \varphi\nabla_j\varphi$ satisfies
\[
\begin{split}
(\partial_{t} - \Box)\theta &= (\varphi^{ij}(h^2)_{ij} + K\Phi)\theta \\
&\quad + (\varphi - \varphi^{ij}h_{ij})|\nabla\varphi|^{2} + 2\varphi^{ij}h^{k}_{i}\nabla_k\varphi\nabla_j\varphi \\
&\quad - \varphi^{kl,ij} (\gamma_{ij}\gamma_{kl}  - 2\alpha_{ij} \gamma_{kl}) \\
&\quad - 2b^{il} \varphi^{jk} \left(\gamma_{ij} \gamma_{kl} - 2\alpha_{ij} \gamma_{kl} + \nabla^2_{ij}\varphi\nabla^2_{kl}\varphi\right) \\
&\quad + KR_{\theta},
\end{split}
\]
\end{lemma}
where
\[
R_{\theta} = -(\varphi^{kl}h_{kl} + \varphi)b^{ir}b^{j}_{r}\nabla_i \varphi\nabla_j\varphi + 2 b^{j}_{k}\varphi^{kl}\nabla_l\varphi\nabla_j\varphi + 2 \varphi\varphi^{ij,kl} g_{ij} \gamma_{kl}.
\]

\begin{proof}
Again using the product rule for \(\Box\), \cref{eq:productbox} and the symmetry \(b^{ij} = b^{ji}\), we have
\begin{equation}
\label{eq:delt_theta}
\begin{split}
(\partial_{t} - \Box)\theta &= (\partial_{t}b^{ij} - \Box b^{ij})\nabla_i \varphi\nabla_j\varphi + b^{ij} (\partial_{t} - \Box) (\nabla\varphi \otimes \nabla\varphi)_{ij} \\
&\quad - 2 \varphi^{kl} \nabla_k b^{ij} \nabla_l (\nabla \varphi \otimes \nabla\varphi)_{ij} \\
&= (\partial_{t}b^{ij} - \Box b^{ij})\nabla_i \varphi\nabla_j\varphi + 2 b^{ij} (\partial_{t} - \Box) (\nabla\varphi)_i \nabla_j\varphi - 2 b^{ij} \varphi^{kl} \nabla^2_{i,k} \varphi \nabla^2_{l,j} \varphi \\
&\quad - 4 \varphi^{kl} \nabla_k b^{ij} \nabla^2_{i,l} \varphi \nabla_j\varphi \\
&= (\partial_{t}b^{ij} - \Box b^{ij})\nabla_i \varphi\nabla_j\varphi + 2 b^{ij} (\partial_{t} - \Box) (\nabla\varphi)_i \nabla_j\varphi \\
&\quad - 2 b^{ij} \varphi^{kl} \nabla^2_{i,k} \varphi \nabla^2_{l,j} \varphi + 4 \varphi^{kl} b^{ip}b^{jq} \nabla_k h_{pq} \nabla^2_{i,l} \varphi \nabla_j\varphi \\
&= (\partial_{t}b^{ij} - \Box b^{ij})\nabla_i \varphi\nabla_j\varphi + 2 b^{ij} (\partial_{t} - \Box) (\nabla\varphi)_i \nabla_j\varphi \\
&\quad - 2 b^{ij} \varphi^{kl} \nabla^2_{i,k} \varphi \nabla^2_{l,j} \varphi + 4 \varphi^{kl} b^{ip}\gamma_{pk} \nabla^2_{i,l} \varphi,
\end{split}
\end{equation}
where in the second to last equality we used the formula for the derivative of the inverse \(b^{ij}\) of \(h_{ij}\) and the Codazzi equation in the last line, producing \(b^{jq} \nabla_k h_{pq} \nabla_j \varphi = b^{jq} \nabla_q h_{pk} \nabla_j \varphi = \gamma_{pk}\). The first term in final line appears on the second to last line of the statement of the lemma (with indices relabelled). The second term is part of \(4 b^{il}\varphi^{jk} \alpha_{ij} \gamma_{kl}\) in the second to last line. So we must deal with the first two terms and show they add to the remainder of the statement. For the first term, we use the evolution of \(b^{ij}\) from \cref{lem:evolution}, \cref{eq:delt_inversesff} to calculate
\begin{equation}
\label{eq:delt_theta1}
\begin{split}
(\partial_{t}b^{ij} - \Box b^{ij})\nabla_i \varphi\nabla_j\varphi &= \nabla_i \varphi \nabla_j \varphi\left[-\varphi^{rs} (h^2)_{rs} b^{ij} + (\varphi^{kl}h_{kl} + \varphi) g^{ij} \right. \\
& \quad - \left(2b^{lq}\varphi^{kp} + \varphi^{kl,pq}\right) b^{ir}b^{js} \nabla_r h_{kl} \nabla_s h_{pq} \\
& \quad - \left. K \{(\varphi + \varphi^{kl}h_{kl}) b^{ir}b^{j}_{r} - \Phi b^{ij}\}\right] \\
&= \left(K\Phi - \varphi^{rs}(h)^2_{rs}\right) \theta - (2b^{lq}\varphi^{kp} + \varphi^{kl,pq}) b^{ir}b^{js}\nabla_i\varphi\nabla_j\varphi\nabla_rh_{kl}\nabla_s h_{pq} \\
&\quad + (\varphi^{kl}h_{kl} + \varphi)|\nabla\varphi|^{2} \\
&\quad - K(\varphi^{kl}h_{kl} + \varphi)b^{ir}b^{j}_{r}\nabla_i \varphi\nabla_j\varphi \\
&= \left(K\Phi - \varphi^{rs}(h)^2_{rs}\right) \theta \\
&\quad + (\varphi^{kl}h_{kl} + \varphi)|\nabla\varphi|^{2} \\
&\quad - \varphi^{kl,pq} \gamma_{kl} \gamma_{pq} \\
&\quad - 2b^{lq}\varphi^{kp} \gamma_{kl} \gamma_{pq} \\
&\quad - K(\varphi^{kl}h_{kl} + \varphi)b^{ir}b^{j}_{r}\nabla_i \varphi\nabla_j\varphi.
\end{split}
\end{equation}
For the second term, from the evolution of \(\nabla\varphi\) in \cref{lem:Evgradphi}, we have
\begin{equation}
\label{eq:delt_theta2}
\begin{split}
2 b^{ij} (\partial_{t} - \Box) (\nabla\varphi)_i \nabla_j\varphi &= 2 b^{ij} \nabla_j\varphi \left[\varphi^{kl,rs}\nabla_i h_{rs} \alpha_{kl}\varphi + 2 \varphi^{kl} b^{rs} \varphi(h^2)_{rl} \nabla_i h_{ks} \right. \\
&\quad + \varphi^{kl}(h^2)_{kl}\nabla_i \varphi + \left(\varphi^{kl}h^{m}_{l}h_{ki} - \varphi^{kl}h_{kl}h^{m}_{i}\right) \nabla_m \varphi\\
&\quad \left. + K\left(\varphi^{kl}g_{ki} \nabla_l \varphi + \varphi \nabla_i \Phi\right)\right] \\
&= 2 b^{ij} \nabla_j\varphi \varphi^{kl}(h^2)_{kl}\nabla_i \varphi \\
&\quad - 2 b^{ij} \nabla_j\varphi \varphi^{kl}h_{kl}h^{m}_{i} \nabla_m \varphi + 2 b^{ij} \nabla_j\varphi \varphi^{kl}h^{m}_{l}h_{ki} \nabla_m \varphi \\
&\quad + 2 b^{ij} \nabla_j\varphi \varphi^{kl,rs}\nabla_i h_{rs} \alpha_{kl} \\
&\quad + 4 b^{ij} \nabla_j\varphi \varphi^{kl} b^{rs} \varphi(h^2)_{rl} \nabla_i h_{ks} \\
&\quad + K\left(2 b^{ij} \nabla_j\varphi \varphi^{kl}g_{ki} \nabla_l \varphi + 2 b^{ij} \nabla_j\varphi \varphi \nabla_i \Phi\right) \\
&= 2 \varphi^{kl}(h^2)_{kl}\theta \\
&\quad - 2 \varphi^{kl}h_{kl} |\nabla\varphi|^2 + 2 \varphi^{kl} h^{m}_{l} \nabla_k\varphi \nabla_m \varphi \\
&\quad + 2 \varphi^{kl,rs} \gamma_{rs} \alpha_{kl} \\
&\quad + 4 b^{rs} \varphi^{kl} \gamma_{ks} \varphi(h^2)_{rl} \\
&\quad + K\left(2 \varphi^{kl} b^j_k \nabla_j\varphi \nabla_l \varphi + 2 \varphi b^{ij} \nabla_j\varphi \nabla_i \Phi\right)
\end{split}
\end{equation}
using the definitions of \(\theta, \alpha_{ij}\) and \(\gamma_{ij}\) as well as \(b^{ij}h_{ki} \nabla_j \varphi = \delta^j_k \nabla_j \varphi = \nabla_k \varphi\), and \(b^{ij} h^m_i = b^{ij} g^{mp}h_{pi} = \delta^j_p g^{mp} = g^{mj}\) in the last equality.

The proof is now completed by adding \cref{eq:delt_theta1} and \cref{eq:delt_theta2} line by line and adding also the final line from \cref{eq:delt_theta}.
\end{proof}
\section{Main evolution equations}
We start this section by calculating the evolution equations of $\chi_2$ and its slight modification, $\chi_3$, which will be employed to obtain Harnack estimates for flow by powers of the mean curvature. We will then focus on the evolution equation of $\chi_1$ which will enable us to deduce (weak) Harnack estimates for all 1-homogeneous convex speeds.
\begin{lemma}
\label{thm:Evchi}
Let $\delta \neq 0.$ The quantity
$
\chi_2 = t(\beta - \theta) + \delta\varphi
$
satisfies
\eq{\label{thm:Evchi1}
\partial_t \chi_2 -\Box\chi_2 &= \left(\frac{\beta - \theta}{\delta\varphi} + \varphi^{ij}(h^2)_{ij} + K\Phi\right)\chi_2 \\
& \quad + t\left(\varphi^{ij,kl} + 2b^{il}\varphi^{jk} - \frac{\varphi^{ij}\varphi^{kl}}{\delta\varphi}\right)\eta_{ij}\eta_{kl} + tK R,
}
where
\[
\eta_{ij} = \alpha_{ij} - \gamma_{ij} = \nabla^2_{i,j}\varphi + (h^2)_{ij}\varphi - b^{rs}\nabla_r h_{ij}\nabla_s \varphi
\]
and
\[
\begin{split}
R &= R_{\beta} - R_{\theta} \\
&= \varphi \Box \Phi + 2\varphi^{kl} \nabla_k \Phi \nabla_l \varphi-\varphi \varphi^{ij,kl}g_{kl} \alpha_{ij} + 2\varphi \varphi^{ij,kl}g_{kl} \eta_{ij}  \\
&\quad + 2\varphi^{2}\varphi^{ij}h_{ij} +(\varphi^{kl}h_{kl} + \varphi)b^{ir}b^{j}_{r}\nabla_i \varphi\nabla_j\varphi - 2 b^{j}_{k}\varphi^{kl}\nabla_l\varphi\nabla_j\varphi.
\end{split}
\]

\end{lemma}

\begin{proof}
We have
\[
(\partial_t - \Box)\chi_2 = \beta - \theta + t(\partial_{t} - \Box)(\beta - \theta) + \delta(\partial_t \varphi - \Box\varphi).
\]
First of all, the evolution equation for \(\varphi\), \cref{lem:evolution}, \cref{eq:delt_speed} gives us
\[
\delta(\partial_t \varphi - \Box\varphi) = \left(\varphi^{ij}(h^2)_{ij} + K\varphi^{ij}g_{ij}\right)\delta\varphi.
\]
Next, we note that
\[
\varphi^{ij} \eta_{ij} = \beta - \theta
\]
since \(\nabla_r \varphi = \varphi^{ij} \nabla_r h_{ij}\). Putting the two equations above together gives
\begin{equation}
\label{eq:deltchi1}
\begin{split}
\beta - \theta + \delta(\partial_t \varphi - \Box\varphi) &= \left(\frac{\beta-\theta}{\delta\varphi} + \varphi^{ij}(h^2)_{ij} + K\varphi^{ij}g_{ij}\right)\delta\varphi + t \frac{(\beta - \theta)^2}{\delta\varphi} - t \frac{(\varphi^{ij}\eta_{ij})^2}{\delta\varphi} \\
&= \frac{\beta-\theta}{\delta\varphi} \chi_2 + \left(\varphi^{ij}(h^2)_{ij} + K\varphi^{ij}g_{ij}\right)\delta\varphi \\
&\quad - t \frac{\varphi^{ij}\varphi^{kl}}{\delta\varphi} \eta_{ij}\eta_{kl}.
\end{split}
\end{equation}

The remaining term \(t(\partial_{t} - \Box)(\beta - \theta)\) is now just bookkeeping. Recall, \cref{lem:evbeta} states that
\begin{align*}
(\partial_{t} - \Box)\beta &= \left(\varphi^{ij}(h^2)_{ij} + K\Phi \right)\beta  & (A) \\
&\quad + (\varphi - \varphi^{ij}h_{ij}) |\nabla\varphi|^{2} + 2\varphi^{ij}h^{k}_{i}\nabla_k \varphi \nabla_j \varphi  & (B) \\
&\quad + \varphi^{ij,kl} \alpha_{ij} \alpha_{kl} & (C) \\
&\quad + 2b^{il}\varphi^{jk} (2\nabla^2_{ij}\varphi\varphi(h^2)_{kl} + \varphi(h^2)_{ij}\varphi(h^2)_{kl}) & (D) \\
&\quad + KR_{\beta}  & (E) \\
\intertext{while \cref{lem:Evtheta} states that}
(\partial_{t} - \Box)\theta &= (\varphi^{ij}(h^2)_{ij} + K\Phi)\theta & (A') \\
&\quad + (\varphi - \varphi^{ij}h_{ij})|\nabla\varphi|^{2} + 2\varphi^{ij}h^{k}_{i}\nabla_k\varphi\nabla_j\varphi & (B') \\
&\quad - \varphi^{kl,ij} (\gamma_{ij}\gamma_{kl}  - 2\alpha_{ij} \gamma_{kl}) & (C') \\
&\quad - 2b^{il} \varphi^{jk} \left(\gamma_{ij} \gamma_{kl} - 2\alpha_{ij} \gamma_{kl} + \nabla^2_{ij}\varphi\nabla^2_{kl}\varphi\right) & (D') \\
&\quad + KR_{\theta}. & (E')
\end{align*}

Subtracting line by line, we have
\begin{align*}
(A) - (A') &= \left(\varphi^{ij}(h^2)_{ij} + K\Phi \right)\beta - \left(\varphi^{ij}(h^2)_{ij} + K\Phi \right)\theta = \left(\varphi^{ij}(h^2)_{ij} + K\Phi \right)(\beta - \theta) \\
(B) - (B') &= (\varphi - \varphi^{ij}h_{ij}) |\nabla\varphi|^{2} + 2\varphi^{ij}h^{k}_{i}\nabla_k \varphi \nabla_j \varphi - (\varphi - \varphi^{ij}h_{ij})|\nabla\varphi|^{2} - 2\varphi^{ij}h^{k}_{i}\nabla_k\varphi\nabla_j\varphi = 0 \\
(C) - (C') &= \varphi^{ij,kl} \alpha_{ij} \alpha_{kl} + \varphi^{kl,ij} (\gamma_{ij}\gamma_{kl}  - 2\alpha_{ij} \gamma_{kl}) = \varphi^{ij,kl} (\alpha_{ij} - \gamma_{ij}) (\alpha_{kl} - \gamma_{kl}) = \varphi^{ij,kl} \eta_{ij} \eta_{kl} \\
(D) - (D') &= 2b^{il}\varphi^{jk} (2\nabla^2_{ij}\varphi\varphi(h^2)_{kl} + \varphi(h^2)_{ij}\varphi(h^2)_{kl}) \\
&\quad + 2b^{il} \varphi^{jk} \left(\gamma_{ij} \gamma_{kl} - 2\alpha_{ij} \gamma_{kl} + \nabla^2_{ij}\varphi\nabla^2_{kl}\varphi\right) \\
&= 2b^{il}\varphi^{jk} \left(\nabla^2_{ij}\varphi\nabla^2_{kl}\varphi +2\nabla^2_{ij}\varphi\varphi(h^2)_{kl} + \varphi(h^2)_{ij}\varphi(h^2)_{kl} - 2 \alpha_{ij} \gamma_{kl} + \gamma_{ij} \gamma_{kl} \right) \\
&= 2b^{il}\varphi^{jk} \left(\alpha_{ij}\alpha_{kl} - 2 \alpha_{ij} \gamma_{kl} + \gamma_{ij} \gamma_{kl} \right) = 2b^{il}\varphi^{jk} \eta_{ij} \eta_{kl} \\
(E) - (E') &= K(R_{\beta} - R_{\theta}) = KR.\\
\end{align*}

Multiplying everything by \(t\) and adding the result to \cref{eq:deltchi1} gives the result.
\end{proof}

We need two more lemmas to obtain a Harnack inequality for $H^{\a}$-flow with $0<\a<1.$  We start by rewriting the term $R$ in the evolution of \(\chi_2\) when the speed is a function of the mean curvature.

\begin{lemma}\label{RSphere}
Suppose that $\p=\p(H).$ Then the term $R$ in the evolution equation of $\chi_2$ takes the form
\eq{\label{RSphere1}
R&=2n\fr{\p''\p}{\p'}\br{\b-\t}-n\fr{\p''\p^2}{\p'}\p^{ij}(h^2)_{ij}+2\p^2\p'H\\
    &\hp{=}+n\br{2\fr{\p''}{\p'}-\fr{\p''^{2}\p}{\p'^{3}}+\fr{\p'''\p}{\p'^{2}}}\p^{ij}\nabla_{i}\p\nabla_j\p\\
    &\hp{=}+\br{\p'H+\p}b^{ir}b^{j}_{r}\nabla_i\p\nabla_j\p-2\p'b^{ij}\nabla_i\p\nabla_j\p.
}
\end{lemma}

\pf{
This computation is tedious but straightforward. The key point is to replace the terms containing second covariant derivatives of $\p,$ namely $\Box\Phi$ and $\a_{ij},$ by a term involving $\Box\p,$ which can then be replaced by $\b-\t$ and some curvature terms.
}

To obtain a Harnack estimate for $H^{\a}$-flow, we will have to handle the middle term in \cref{RSphere1}; this term does not always have the favorable positive sign. To this aim, it is useful to add an auxiliary function of the speed. Using \cref{thm:Evchi} and \cref{RSphere}, it is straightforward to obtain the following evolution equation for $\chi_3=\chi_2+t\zeta$, where $\zeta=\zeta(\p)$ is a function of $\p.$

\begin{lemma}\label{lemma : lem9}
The quantity
\eq{\chi_3=\chi_2+t\zeta}
evolves according to
\eq{\label{Evchibar1}
\del_t \chi_3 -\Box\chi_3 &= \left(\fr{\b - \t}{\d\p} + \p^{ij}(h^2)_{ij} + \p^{ij}g_{ij}\right)\chi_2+\zeta \\
&\hp{=} + t\left(\p^{ij,kl} + 2b^{il}\p^{jk} - \fr{\p^{ij}\p^{kl}}{\d\p}\right)\eta_{ij}\eta_{kl}\\
        &\hp{=}+t\Big[2n\fr{\p''\p}{\p'}\br{\b-\t}+\br{\zeta'-n\fr{\p''\p}{\p'}}\p^{ij}(h^2)_{ij}\p+\zeta'\p^{ij}g_{ij}\p\\
        &\hp{=+t}+2\p^2\p'H+\br{n\br{2\fr{\p''}{\p'}-\fr{\p''^{2}\p}{\p'^{3}}+\fr{\p'''\p}{\p'^{2}}}-\zeta''}\p^{ij}\nabla_{i}\p\nabla_j\p\\
    &\hp{=+t}+\br{\p'H+\p}b^{ir}b^{j}_{r}\nabla_i\p\nabla_j\p-2\p'b^{ij}\nabla_i\p\nabla_j\p\Big].
}
\end{lemma}
Lemmas \ref{thm:Evchi}, \ref{RSphere}, and \ref{lemma : lem9} enable us to get a strong Harnack estimate for $H^{\a}$-flow; see Section \ref{Harnack} and Theorem \ref{thm: main 1}. Due to the presence of $\Box\Phi$ in $R$ given in \cref{thm:Evchi}, it is not clear to us whether $\chi_2$ would results in Harnack estimates for curvature flows other than $H^{\a}$-flow. As it will be shown, by weakening $\chi_2$ to $\chi_1=\chi_2+tK\p\Phi$,  we can obtain (weak) Harnack estimates for all 1-homogeneous convex speeds.

\begin{lemma}\label{WeakHarnackEv}
The quantity
\eq{\chi_1=t(\del_t\p-\t)+\d\p}
satisfies the evolution equation
\eq{\label{WeakHarnackEv1}\del_t\chi_1-\Box\chi_1&=\br{\fr{\b-\t}{\d\p}+\p^{ij}(h^2)_{ij}+K\Phi}\chi_1-\fr{tK}{\d}\Phi(\b-\t)+K\Phi\p\\
                &+t\p^{ij,kl}\br{\eta_{ij}+K\p g_{ij}}\br{\eta_{kl}+K\p g_{kl}}+t\br{2b^{il}\p^{jk}-\fr{\p^{ij}\p^{kl}}{\d\p}}\eta_{ij}\eta_{kl}\\
                &+tK\br{2\p^2\p^{ij}h_{ij}+\br{\p^{ij}h_{ij}+\p}b^{ir}b^{j}_{r}\nabla_{i}\p\nabla_{j}\p-2b^j_k\p^{kl}\nabla_{l}\p\nabla_{j}\p}.
}
We may also rewrite
$-\fr{tK}{\d}\Phi(\b-\t)+K\Phi\p=\fr{K\Phi}{\d}(-\chi_1+tK\p\Phi+2\d\p).$
This in turn implies that  $\chi_1$ satisfies the evolution equation
\eq{\label{WeakHarnackEv12}\del_t\chi_1-\Box\chi_1&=\br{\fr{\b-\t}{\d\p}+\p^{ij}(h^2)_{ij}+\left(1-\fr{1}{\d}\right)K\Phi}\chi_1+\fr{K\Phi\p}{\d}(tK\Phi+2\d)\\
                &+t\p^{ij,kl}\br{\eta_{ij}+K\p g_{ij}}\br{\eta_{kl}+K\p g_{kl}}+t\br{2b^{il}\p^{jk}-\fr{\p^{ij}\p^{kl}}{\d\p}}\eta_{ij}\eta_{kl}\\
 &+tK\left(2\p^2\p^{ij}h_{ij}+\left[\br{\p^{ij}h_{ij}+\p}b^{ir}-2\p^{ir}\right]b^{j}_{r}\nabla_{i}\p\nabla_{j}\p\right).
}

\end{lemma}

\pf{
We simply use the evolution of $\chi_2,$ cf.~\eqref{thm:Evchi1}, and add the evolution of $tK\Phi\p.$ This is
\eq{\label{WeakHarnackEv2}\br{\del_t-\Box}\br{tK\Phi\p}&=tK\Big(\Phi\p\p^{ij}(h^2)_{ij}+K\Phi^2\p+\p\del_t\Phi\\
                    &\hp{-tk\Big(}-\p\Box\Phi-2\p^{ij}\nabla_{i}\p\nabla_{j}\Phi\Big)\\
                            &\hp{=}+K\Phi\p.
                            }
Noting that
\eq{\del_t\Phi=\del_t\br{\p^{ij}g_{ij}}=\p^{ij,kl}\br{\a_{kl}+K\p g_{kl}}g_{ij}}
and adding \eqref{WeakHarnackEv2} to the term $R$ in \eqref{thm:Evchi1}, we obtain the claim.
}


\section{Harnack Inequalities}\label{Harnack}

\subsection{Euclidean Space}

In Euclidean space, we deduce differential Harnack inequalities for various speeds, recovering well known Harnack inequalities \cite{MR1296393, MR1100812, MR1316556, MR2813400, MR1480081}.

\begin{corollary}\label{EucHarnack}
Let $x$ be a strictly convex solution of \eqref{eq:CurvFlow} in Euclidean space. Suppose $f$ is inverse concave and the pair $\p\in C^{\8}(\R_+),~\d\neq 0$ satisfy
\eq{\p''+\fr{2\p'}{f}-\fr{1}{\d}\fr{\p'^2}{\p}\geq 0,\quad \d\p(0,\xi)>0.}
The non-negativity of the function
\eq{\chi_2=t(\partial_t\p-b^{ij}\nabla_i\p\nabla_j\p)+\d\p}
is preserved along \eqref{eq:CurvFlow}.
\end{corollary}

\pf{
In \eqref{thm:Evchi1} all that has to be checked is non-negativity of the bracket involving the second derivative of $\p(f).$
First note that the curvature function $f$ is inverse concave, i.e.,
\eq{\label{eq: inverse concavity}\~f(\k_i)=\fr{1}{f(\k_i^{-1})}}
is concave, if and only if
\eq{\label{InvConcave}\left(f^{ij,kl}+2f^{ik}b^{jl}-2\fr{f^{ij}f^{kl}}{f}\right)\eta_{ij}\eta_{kl}\geq0
}
for all symmetric $\eta.$
We calculate
\eq{
\left(\p^{ij,kl}+2b^{jl}\p^{ik}-\fr{\p^{ij}\p^{kl}}{\d\p}\right)\eta_{ij}\eta_{kl}&=\left(\p'f^{ij,kl}+\p''f^{ij}f^{kl}+2\p'f^{ik}b^{jl}-\fr{\p'^2}{\d\p}\right)(f^{ij}\eta_{ij})^2\\
                        &\geq \br{\p''+2\fr{\p'}{f}-\fr{\p'^2}{\d\p}}(f^{ij}\eta_{ij})^2\\
                        &\geq 0
}
in the sense of a bilinear form on symmetric matrices.
}

Easy calculations imply that the assumptions of Corollary \ref{EucHarnack} are satisfied in quite a wide range of interesting situation.

\begin{corollary}\label{HarnackEucCor}
 If $f$ is an inverse concave curvature function, then for contracting flows the pair
\eq{\p(f)=f^{\a},\quad 0<\a<\8, \quad \d\geq \fr{\a}{\a+1},}
and for expanding flows the pair
\eq{\p(f)=-f^{-\b}, \quad 0<\b<1,\quad \d\leq \fr{\b}{\b-1}}
satisfy $$\partial_t \p-b^{ij}\nabla_i\p\nabla_j\p+\frac{\d\p}{t}>0.$$
%If $f$ is inverse convex (e.q., the inequality (\ref{eq: inverse concavity}) is reversed), then for expanding flows the pair
%\eq{\p(f)=-f^{-\b}, \quad \b>1,\quad \d\geq \fr{\b}{\b-1}}
%satisfy $\chi<0.$
\end{corollary}

\subsection{The Sphere}

\begin{theorem}\label{thm: main A}
Suppose $f$ is a  convex curvature function. If $\p=f$,then for contracting flows
$$\partial_t \p-b^{ij}\nabla_i\p\nabla_j\p+\frac{n}{2}\frac{\p}{t}>0.$$
\end{theorem}
\begin{proof}
In view of the maximum principle and that $\chi_1$ is manifestly positive at $t=0$, it suffices to show that the right-hand side of (\ref{WeakHarnackEv12}) is positive whenever at some point in space-time $\chi_1=0$. By convexity of $\p$, the term with the coefficient $\p^{ij,kl}$ is positive. On the other hand, note that any 1-homogeneous curvature function $\p$ satisfies $\p b^{ij}\geq \p^{ij}.$ Therefore
$$\left(\br{\p^{ij}h_{ij}+\p}b^{ir}-2\p^{ir}\right)b^{j}_{r}\nabla_{i}\p\nabla_{j}\p\ge 0.$$ To complete the proof note that
$$\br{2b^{il}\p^{jk}-\fr{\p^{ij}\p^{kl}}{\d\p}}\eta_{ij}\eta_{kl}\geq \br{\frac{2\p^{il}\p^{jk}}{\p}-\fr{\p^{ij}\p^{kl}}{\d\p}}\eta_{ij}\eta_{kl}\geq \left(\frac{2}{n}-\frac{1}{\delta}\right)\frac{(\p^{ij}\eta_{ij})^2}{\p}.$$
\end{proof}
\begin{remark}
For power means $H_r=\left(\sum \kappa_i^r\right)^{\frac{1}{r}}$ and $r\geq 1,$ it is not difficult, using AM-GM inequality, to see that $\br{2b^{il}\p^{jk}-\fr{\p^{ij}\p^{kl}}{\d\p}}\eta_{ij}\eta_{kl}\geq0$ with $\delta=\frac{1}{2};$ therefore, the contracting flow with speed $\p=H_r$ and $r\geq 1$ satisfies
$$\partial_t H_r-b^{ij}\nabla_iH_r\nabla_jH_r+\frac{H_r}{2t}>0.$$
\end{remark}

Employing the evolution \cref{Evchibar1}, we can obtain a stronger Harnack inequality for the speed \(\p = H^{\alpha}\) with \(\alpha \in (0,1)\); the case $\alpha=1$ was considered in \cite{2015arXiv150802821B}.
\begin{theorem}\label{thm: main 1}
If $\frac{1}{2}+\frac{1}{2n}\leq {\alpha}< 1,$ then
\[
\partial_t H^{\alpha} - b^{ij}\nabla_iH^{\alpha}\nabla_jH^{\alpha} - \frac{K {\alpha}}{2{\alpha}-1}H^{2{\alpha}-1} + \frac{{\alpha}}{{\alpha}+1} \frac{H^{\alpha}}{t} > 0.
\]
If $0<{\alpha}\leq \frac{1}{2} + \frac{1}{2n}$ or $\alpha=1$, then

\[
\partial_t H^{\alpha} - b^{ij}\nabla_iH^{\alpha}\nabla_jH^{\alpha} - K n{\alpha}H^{2{\alpha}-1} + \frac{{\alpha}}{{\alpha}+1} \frac{H^{\alpha}}{t} > 0.
\]
\end{theorem}

\begin{proof}
In order to prove Theorem \ref{thm: main 1}, we need to show that for
\eq{\p=H^{\a},\quad \d=\fr{\a}{\a+1},}
the quantity $\chi_3$ preserves its positivity at all $t>0.$ Here $\zeta$ is chosen to be
\eq{\zeta(\p)=\begin{cases} 0, &0<\a\leq \fr{1}{2}+\fr{1}{2n}\\
                    \a\br{n-\fr{1}{2\a-1}}\p^{2-\fr{1}{\a}}, &\fr{1}{2}+\fr{1}{2n}<\a< 1.\end{cases}}
However, to avoid confusion, we will keep the general form as long as possible.

At time $t=0,$ $\chi_3$ is positive.
%%%%%%%%%%%% We need to impose strict mean convexity in the main theorem here!!!!!!! %%%%%%
Thus suppose there exists a first time $t_0$ and a point $x_0$ in $M_{t_0},$ such that $\chi_3(t_0,x_0)=0.$
Then we also obtain
\eq{\chi_2=-t\zeta,\quad \b-\t=-\fr{\d\p}{t}-\zeta.}
Thus, using \eqref{Evchibar1}, we obtain at $(t_0,x_0):$
\eq{0&\geq \del_t\chi_3-\Box\chi_3\\
        &=2\zeta-2n\d\fr{\p''\p^2}{\p'}+t\br{\p^{ij,kl}+2b^{il}\p^{jk}-\fr{\p^{ij}\p^{kl}}{\d\p}}\eta_{ij}\eta_{kl}\\
        &\hp{=}+t\Big[\fr{\zeta^2}{\d\p}-2n\fr{\p''\p}{\p'}\zeta+2\p^2\p'H+\br{\zeta'\p-\zeta}\p^{ij}g_{ij}    \\
        &\hp{=+t} +\br{\zeta'\p-n\fr{\p''\p^2}{\p'}-\zeta}\p^{ij}(h^2)_{ij}+\br{\p'H+\p}b^{ir}b^j_r\nabla_i\p\nabla_j\p\\
        &\hp{=+t}-2\p'b^{ij}\nabla_i\p\nabla_j\p+\br{n\br{2\fr{\p''}{\p'}-\fr{\p''^{2}\p}{\p'^{3}}+\fr{\p'''\p}{\p'^{2}}}-\zeta''}\p^{ij}\nabla_{i}\p\nabla_j\p\Big]\\
        &\geq2\zeta-2n\d\fr{\p''\p^2}{\p'}+t\br{\p^{ij,kl}+2b^{il}\p^{jk}-\fr{\p^{ij}\p^{kl}}{\d\p}}\eta_{ij}\eta_{kl}\\
        &\hp{=}+t\Big[\fr{\zeta^2}{\d\p}+2\p^2\p'H-2n\fr{\p''\p}{\p'}\zeta+n\br{\zeta'\p-\zeta}\p'    \\
        &\hp{=+t} +\br{\zeta'\p-n\fr{\p''\p^2}{\p'}-\zeta}\p^{ij}(h^2)_{ij}\\
        &\hp{=+t}+\br{n\br{2\fr{\p''}{\p'}-\fr{\p''^{2}\p}{\p'^{3}}+\fr{\p'''\p}{\p'^{2}}}-\zeta''+\fr{\p}{\p'H^2}-\fr{1}{H}}\p^{ij}\nabla_{i}\p\nabla_j\p\Big],}

where in the last inequality we used
\eq{\p-\p'H=(1-\a)H^\a\geq 0}
and
\eq{Hb^{ij}\geq g^{ij}}
in the sense of bilinear forms.

To finish the proof, we need to show that the right-hand side is positive. If $\zeta=0$, this is straightforward:
\[\varphi''<0~\&~\br{n\br{2\fr{\p''}{\p'}-\fr{\p''^{2}\p}{\p'^{3}}+\fr{\p'''\p}{\p'^{2}}}+\fr{\p}{\p'H^2}-\fr{1}{H}}\geq 0.\]
For the second case that $\zeta\neq 0$, note that
$$2\zeta-2n\d\fr{\p''\p^2}{\p'}\geq 0~\mbox{for}~\alpha\geq \frac{n+1}{2n},$$
$$\br{\p^{ij,kl}+2b^{il}\p^{jk}-\fr{\p^{ij}\p^{kl}}{\d\p}}\eta_{ij}\eta_{kl}\geq0~\mbox{for}~\delta=\frac{\alpha}{\alpha+1},$$
$$\fr{\zeta^2}{\d\p}-2n\fr{\p''\p}{\p'}\zeta+n\br{\zeta'\p-\zeta}\p'\geq 0~\mbox{for}~\alpha\geq\frac{n+1}{2n}~\&~\delta=\frac{\alpha}{\alpha+1},$$
$$\br{\zeta'\p-n\fr{\p''\p^2}{\p'}-\zeta}\geq 0~\mbox{for}~\frac{n+1}{2n}\leq \alpha\leq 1,$$
$$\br{n\br{2\fr{\p''}{\p'}-\fr{\p''^{2}\p}{\p'^{3}}+\fr{\p'''\p}{\p'^{2}}}-\zeta''+\fr{\p}{\p'H^2}-\fr{1}{H}}=0.$$
\end{proof}

\section{Preserving convexity}

In the derivation of the Harnack inequalities we have assumed the strict convexity of flow hypersurfaces. In this section, we show that strict convexity is preserved by a variety of flow speeds.

\subsection{Flows in the sphere}
In the sphere, we obtain the preservation of strict convexity for all flows for which we could prove the Harnack inequality, cf.~Theorem \ref{thm: main A} and Theorem \ref{thm: main 1}.

\begin{proposition}
 Let $M_{0}$ be a closed and strictly convex initial hypersurface for the curvature flow equation
\eq{\dot{x}=-\p\nu}
in the sphere. Suppose either $\p=f$ with $f$ as a convex curvature function, or  $\p=H^{\a}$ and $\a\in (0,1)$.
Then all flow hypersurfaces are strictly convex.
\end{proposition}

\pf{
Let us first treat the case $\p=H^{\a}$ and $0<\a<1.$ The inverse curvature function of $H$ is the harmonic mean curvature
\eq{\~H=\br{\sum_{i=1}^n\k_i^{-1}}^{-1}.}
Let $T$ be the first time, where the strict convexity is lost. Then on the time interval $[0,T)$ the dual flow defined via the Gauss map is well defined and reads
\eq{\label{DualFlow}\dot{\~x}=\fr{1}{\~H^{\a}}\~\nu,}
where the harmonic mean curvature $\~H$ is now evaluated at $\~\k_i=\k_i^{-1},$ compare \cite{Gerhardt:/2015} for the derivation of the dual flow. As a curvature function, the harmonic mean curvature is 1-homogeneous, strictly monotone, concave and vanishes on the boundary of $\G_{+} $ (regarding the concavity see \cite[Lemma~2.2.12, Lemma~2.2.14]{Gerhardt:/2006}). For flows of the kind \eqref{DualFlow} uniform curvature estimates were deduced in \cite[Lemma~4.7]{MakowskiScheuer:/2013}, implying that the $\~\k_{i}$ are bounded. This means that up to time $T$ uniform convexity is preserved for the original flow, which contradicts the definition of $T,$ if $T$ is not the collapsing time.

Now we treat convex speeds. We use Hamilton's maximum principle for tensors to deduce that the tensor
\eq{S_{ij}=h_{ij}-\e g_{ij}}
remains non-negative, if $\e$ is chosen so that this is the case initially. We use the evolution equations \eqref{eq:delt_metric} and \eqref{eq:delt_sff_box} to deduce
\eq{\label{PresConv}\del_tS_{ij}-\Box S_{ij}&=\p'f^{kl}(h^2)_{kl}h_{ij}-(\p'f+\p)(h^2)_{ij}+K(\p'f+\p)g_{ij}\\
                    &\hp{=}-K\p'f^{kl}g_{kl}h_{ij}+2\e \p h_{ij}+\p^{kl,rs}\nabla_{i}h_{kl}\nabla_{j}h_{rs}\\
                    &=:N_{ij}.}
Using convexity of $f$, for any unit length null eigenvector $\eta$ of $S_{ij}$ we have
\eq{N_{ij}\eta^i\eta^j\geq -2\e^2f+2Kf-\e Kf^{kl}g_{kl}+2\e^2 f.}
On the other hand, for convex and $1$-homogeneous curvature functions there holds
\eq{f^{kl}g_{kl}\leq n,}
cf.~\cite[Lemma~2.2.19]{Gerhardt:/2006}. Furthermore, $S_{ij}$ is still positive semi-definite and thus $f\geq n\e.$
Hence
\eq{N_{ij}\eta^i\eta^j\geq n\e>0.}
}

\subsection{Flows in the Euclidean space}
In Euclidean space the question of preserved convexity has been addressed more thoroughly. It is also known that there is a variety of examples where convexity is lost for contracting flows; see \cite{AndrewsMcCoyZheng:07/2013}. In this paper, the authors also discuss necessary and sufficient conditions to conclude preserved convexity. In other special situations preserved convexity was proved, e.g., see \cite{Andrews:/1994b}, \cite{Andrews:/2007}, \cite{Andrews:04/2010} and \cite{Schulze:/2005}.
It does not seem possible to obtain preserved convexity in the generality in which we can prove a Harnack inequality, cf. Corollary \ref{HarnackEucCor}. In the following proposition, we provide a large class of flows, for which convexity is preserved. Some of these cases seem not to be covered in the existing literature.
%%%%%%%We have to find out which ones those are!%%%%%%

\begin{proposition}
Let $M_{0}$ be a closed and strictly convex initial hypersurface for the curvature flow equation
\eq{\dot{x}=-\p\nu}
in the Euclidean space. Suppose $f$ is an inverse concave curvature function. Then in any of the cases
\eq{\p(f)=f^{\a},\quad 0<\a<\8,}
\eq{\p(f)=-f^{-\b},\quad 0<\b\leq 1,}
all flow hypersurfaces remain strictly convex, provided that in the contracting case with $\a>1,$ $f$ is additionally supposed to be convex and in the expanding case $f$ is additionally supposed to be concave.
\end{proposition}

\pf{
(i)~Let us first consider the case $\p=f^{\a},$ $\a\leq 1.$ We use \eqref{eq:delt_inversesff} to compute the evolution equation of the inverse of the Weingarten map,
\eq{b^i_j=b^{ik}g_{kj}.}
We obtain
\eq{\del_t b^i_j&=\Box b^i_j-\p'f^{rs}(h^2)_{rs}b^i_j+(\p'f-\p)\d^i_j\\
				&\hp{=}-\br{\p'f^{kl,pq}+2\p'b^{lq}f^{kp}+\p''f^{kl}f^{pq}}b^{ir}b^{s}_{j}\nabla_rh_{kl}\nabla_sh_{pq}.}
Since $f$ is inverse concave, from \eqref{InvConcave} we deduce
\eq{\p'f^{kl,pq}+2\p'b^{lq}f^{kp}+\p''f^{kl}f^{pq}\geq 2\fr{\p'}{f}f^{ij}f^{kl}+\p''f^{ij}f^{kl}\geq 0}
in the sense of bilinear forms for all $\p$ under consideration.
Since $\a\leq 1,$ all other terms are negative as well and we deduce that in any coordinate system all diagonal elements of $(b^i_j)$ are bounded. Hence all principal curvatures remain bounded from below.

(ii)~If $\a>1$ and $f$ is convex, we use the same proof as in the spherical case to show that the tensor
\eq{S_{ij}=h_{ij}-\e g_{ij}}
remains non-negative. Equation \eqref{PresConv} was already formulated in a generality which allows to consider this new situation. At a first point in spacetime where we have a null-eigenvector of $S_{ij}$ we deduce, also using $S_{ij}\geq 0$ 
\eq{N_{ij}\eta^i\eta^j\geq \e\p'f^{kl}(h^2)_{kl}-\e^2\p'f+\e^2\p\geq \e^2\p>0.}
Here we also used the convexity of the composition $\p(f).$

(iii)~In the expanding case we use Andrews' maximum principle, \cite[Thm.~3.2, Thm.~4.1]{Andrews:/2007} to deduce preserved pinching. Therefore consider the tensor
\eq{T_{ij}=h_{ij}-\e Hg_{ij},}
where $0<\e<\fr 1n$ is small enough to make $T_{ij}$ positive initially.
Using the evolution equations \eqref{eq:delt_metric}, \eqref{eq:delt_sff_box} and \eqref{eq:delt_weingarten_box}, we obtain
\eq{\del_t T_{ij}-\Box T_{ij}&=\p'f^{kl}(h^2)_{kl}T_{ij}-(\p'f+\p)(h^2)_{ij}+\e\br{\p'f-\p}\|A\|^2g_{ij}+2\e H\p h_{ij}\\
							&\hp{=}+\p^{kl,rs}\nabla_ih_{kl}\nabla_jh_{rs}-\e\p^{kl,rs}\nabla_mh_{kl}\nabla^mh_{rs}g_{ij}\\
                            &\equiv N_{ij}+\~N_{ij},}
  where $\~N_{ij}$ represents all term including derivatives of $h_{ij}.$ At a first unit length null-eigenvector of $T_{ij}$ we have
\eq{h_{ij}\eta^i=\e H\eta_j}
and hence
\eq{N_{ij}\eta^i\eta^j&=-\e^2 H^2(\p'f+\p)+\e\br{\p'f-\p}\|A\|^2+2\e^2H^2\p\\
					&>0, }
since $\|A\|^2\geq \tfrac{H^2}{n}.$ To deduce that also $\~N_{ij}\eta^i\eta^j\geq 0$ is true, we have to provide the modified null-eigenvector condition as defined in \cite[Thm.~3.2]{Andrews:/2007}, which holds for terms of the form $\~N_{ij},$ if $\p$ is a symmetric, strictly monotone, concave and inverse concave curvature function, compare \cite[Thm.~4.1]{Andrews:/2007}, where inverse concavity has to be understood as concavity of
\eq{\hat{\p}(\k_i)=-\p(\k_i^{-1}).}
$\p$ is obviously monotone and symmetric. It is concave as a composition of the concave function $x\mt -x^{-\b}$ and the concave curvature function $f.$ The speed $\p$ is also inverse concave, since
\eq{-\p(\k_i^{-1})=f^{-\b}(\k_i^{-1})=\br{\fr{1}{f(\k_i^{-1})}}^{\b},}
which is concave as the composition of concave functions, recall that $\b\leq 1.$
}

\section{Ancient and Quasi Ancient Solutions}

\subsection{Backwards Limit}
We consider the spherical ambient space, $K=1,$ without further mention, and we are interested in solutions with maximal possible lifetime. To understand this maximal time, we define $T_S$ to be the lifespan of \it{the} convex spherical solution of \eqref{eq:CurvFlow}. By the convex spherical solution we mean a family of geodesic spheres shrinking under the flow \eqref{eq:CurvFlow} collapsing to a point at time $t=0$ and existing on the maximal interval \((-T_S, 0)\). For 1-homogeneous $\p$, \(T_S = \infty\), but for $\alpha$-homogeneous $\p$ with $\alpha<1$, \(T_S\) is finite.

\begin{lemma}
 Consider \eqref{eq:CurvFlow} with speed \(\p = H^{\alpha}\) for \(\alpha \in (0,1)\). Then a flow of strictly convex geodesic spheres has a finite lifespan, i.e. let $S_r(p)$ be a geodesic sphere in $\S^{n+1}$ around $p\in \S^{n+1}$. Then the flow exists only for a finite time interval \(-T_S,0\) with \(0 < T_S < \infty\), collapsing to a point at \(t=0\) and converging to an equator at \(t=T_S\).
\end{lemma}

\begin{proof}
Since $H$ is constant on a geodesic sphere, for a spherical flow the evolution equation for $\p=H^{\a}$ yields
\eq{\fr{d}{dt}H^{\a}\geq \a n H^{2\a-1}.}
This yields
\eq{\fr{d}{dt}H\geq nH^{\a}.}
Since the right hand side remains strictly positive under this ODE we obtain finite lifespan forward in time.
Convexity and integration over some interval $(a,b)$ yield
\eq{0\leq H^{1-\a}(a)\leq H^{1-\a}(b)-(1-\a)n(b-a).}
Letting $a\ra-\8$ gives finite existence backwards in time.
\end{proof}

\begin{lemma}
Let $x$ be a convex solution of \eqref{eq:CurvFlow}, defined on the open interval $(-T,0),$ where $0$ is the collapsing time. Then
there holds
\eq{T\leq T_S.}
\end{lemma}

\begin{proof}
Suppose $T>T_S+\e$ for some $\e>0$. Since $M=M_{-T_S-\fr{\e}{2}}$ bounds a convex body $\hat{M}$, it strictly contained in an open hemisphere due to the classical paper \cite{CarmoWarner:/1970}. Then there exists a geodesic sphere $S$ with $\hat{M}\sub\hat{S}.$ By the avoidance principle the flow with initial hypersurface $M$ collapses before the spherical flow contradicting $T>T_S$.
\end{proof}

Due to this lemma the following definition is reasonable.

\begin{definition}
A convex solution of \eqref{eq:CurvFlow} defined on an interval $(-T,0)$ is called \it{quasi-ancient}, if $T=T_S$.
\end{definition}
The term ancient is reserved for the situation when \(T_S=\infty\), and by the definition, ancient solutions are also quasi-ancient.

The aim of this section is to prove that for a quasi-ancient solution of \eqref{eq:CurvFlow} the backwards limit of the flow hypersurfaces \it{with bounded mean curvature}, $M_t$ is an equator for $t\ra -T_S$. We will use the method of \cite{MakowskiScheuer:/2013} to achieve this.
For strictly monotone, 1-homogeneous convex speeds, \cref{cor:boundedH} gives a bound on mean curvature for ancient solutions. For quasi-ancient solutions, the Harnack inequality does not in general give such a bound, since we cannot send \(t \to -\infty\) whenever \(T_S\) is finite. One can envisage backwards limits as convex polyhedra and hence with unbounded \(H\), but it is not clear that these can arise as limits of quasi-ancient solutions. Thus at this stage we must make the additional assumption that \(H\) is bounded for quasi-ancient solutions.
\begin{proposition}
\label{cor:boundedH}
Suppose $f\in C^{\8}(\G_+)$ is a strictly monotone, 1-homogeneous convex curvature function. Then any convex ancient solution of the contracting flow with speed $\p(f)=f$ satisfies
\[\partial_t \p-b^{ij}\nabla_i\p\nabla_j\p\geq 0.\]
In particular, for all $t\le 0$ we have
$H(\cdot,t)\leq c.$
Here $c<\infty$ depends only on $M_0.$
\end{proposition}
\begin{proof}
For any $t>s$, the  Harnack estimate of Theorem \ref{thm: main A} implies that
$$\partial_t \p-b^{ij}\nabla_i\p\nabla_j\p+\frac{n}{2}\frac{\p}{t-s}>0.$$
Allowing $s\to-\infty$ proves the first claim. For the second claim, observe that for any 1-homogeneous convex $f$ we have \[f\ge \frac{f(1,\cdots,1)}{n}H,\]
see \cite[Chapter 2]{Gerhardt:/2006}. Therefore, ancient solutions satisfy
\[H(\cdot,t)\leq \frac{n}{f(1,\cdots,1)}\p(\cdot,t)\leq \frac{n}{f(1,\cdots,1)}\p(\cdot,0). \]
\end{proof}

\begin{lemma}\label{ISC}
Let $x$ be a quasi-ancient solution of \eqref{eq:CurvFlow}. Then there holds:\\

(i)~For all $t_0<0$ there exists a uniform radius $R>0,$ such that the enclosed convex bodies $\hat{M}_t,$ $-T_S<t\leq t_0,$ of the flow hypersurfaces $M_t$ satisfy a uniform interior sphere condition with radius $R.$\\

(ii)~For every $y_0\in\mrm{int}~\hat{M}_{t_0}$ the hypersurfaces $M_t,$ $-T_S<t\leq t_0$ can be written as a graph in geodesic polar coordinates around $y_0$ and the corresponding graph functions satisfy uniform $C^2$-estimates.
\end{lemma}

\pf{
Fix an interior point $y_0\in \mrm{int}~\hat{M}_{t_0}.$ Since for a convex contracting $H^{\a}$ flow the enclosed convex bodies of the flow hypersurfaces are strictly decreasing, they are strictly increasing backwards in time. By \cite[Lemma~3.9]{MakowskiScheuer:/2013} there exists a closed hemisphere $\mc{H}(x_0),$ such that
\eq{\hat{M}_t\sub\mc{H}(x_0).}
In our situation all hypersurfaces $M_t,$ $-T_S<t\leq t_0,$ satisfy
\eq{B_{\e}(y_0)\sub \mrm{int}~\hat{M}_t}
and
\eq{B_{\e}(\hat{y}_0)\sub \hat{M}_t^c}
with a uniform $\e,$ where $\hat{y}_0$ denotes the antipodal point of $y_0.$
Now we prove the two claims.\\

(i)~Consider the stereographic projection with $\hat{y}_0$ corresponding to infinity. The image hypersurfaces are then strictly convex hypersurfaces in the Euclidean space with uniformly bounded second fundamental form. Blaschke's rolling theorem, cf.~\cite{Blaschke:/1956}, gives the interior sphere condition.\\

(ii)~
Write the $M_t$ as graphs in geodesic polar coordinates around $y_0,$
\eq{M_t=\{(r,x^i)\cn r=u(t,x^i)\},}
where $r$ describes the geodesic distance to $y_0.$ In these coordinates the spherical metric takes the form
\eq{d\-s^2=dr^2+\sin^{2}r\s_{ij}dx^idx^j,}
where $(\s_{ij})$ is the round metric of $\S^n.$

Hence on the set in which the $M_t$ range, the metrics $\-g_{ij}=\sin^2r\s_{ij}$ and $\s_{ij}$ are equivalent.
Due to \cite[Thm.~2.7.10]{Gerhardt:/2006} for all convex hypersurface $M_t$ the quantity
\eq{v^2=1+\-g^{ij}\nabla_iu\nabla_ju}
is uniformly bounded by a constant which only depends on $\e.$
Hence by the equivalence of norms the $M_t$ are uniformly $C^1$-bounded in the sense that the corresponding functions $u(t,\cdot)$ are uniformly $C^1(\S^n)$-bounded.
A straightforward computation yields the following representation of the Weingarten map in terms of the function $u,$ namely
\eq{h^i_j=\fr{\vt'}{v\vt}\d^i_j+\fr{\vt'}{v^3\vt^3}\nabla^iu\nabla_ju-\fr{\~g^{ik}}{v\vt^2}\nabla^2_{kj}u,}
where $\~g^{ij}$ is the inverse of $\~g_{ij}=\vt^{-2}g_{ij},$ $\vt(r)=\sin r$ and covariant derivatives as well as index raising is performed with respect to $\s_{ij},$ compare for example \cite[(3.82)]{Scheuer:05/2015}. Due to the curvature estimates we obtain uniform $C^2(\S^n)$-estimates.
}

\begin{corollary}\label{Backlimit}
Let $x$ be a quasi-ancient solution of \eqref{eq:CurvFlow}. Then there exists a unique backwards limiting hypersurface $M_{-T_S}$ and the flow hypersurfaces $M_t$ converge to $M_{-T_S}$ in $C^{1,\b},$ $0<\b<1,$ in the sense that for a common graph representation as in Lemma \ref{ISC} there holds
\eq{u(t,\cdot)\ra u(-T_S,\cdot)}
in the norm of $C^{1,\b}(\S^n).$
\end{corollary}

\pf{
Due to the pointwise monotonicity of $u(t,\cdot)$ backwards in time we obtain a pointwise limit. The $C^{1,\b}$-convergence follows from compactness.
}

\begin{theorem}
In the situation of Corollary \ref{Backlimit} the hypersurface $M_{T_S}$ is an equator of $\S^{n+1}.$
\end{theorem}

\pf{
Since the convex bodies $M_t$ are increasing backwards in time and due to the uniform convergence of $M_t$ to $M_{-T_S},$ the set
\eq{\hat{M}_{-T_S}:=\overline{\bigcup_{t<0}\hat{M}_t}}
is a compact body with
\eq{\del \hat{M}_{-T_S}=M_{-T_S}.}
Since $\mrm{int}(\hat{M}_{-T_S})$ is a strictly convex set, it is especially weakly convex in a hemisphere in the sense of \cite[Def.~3.2]{MakowskiScheuer:/2013}. Thus $\hat{M}_{-T_S}$ is a weakly convex body in a hemisphere. The proof of \cite[Lemma~6.1]{MakowskiScheuer:/2013} can literally be applied to show that $\hat{M}_{-T_S}$ satisfies a uniform interior sphere condition as well.
We can apply \cite[Thm.~1.1]{MakowskiScheuer:/2013} and obtain that $\hat{M}_{-T_S}$ is either strictly contained in an open hemisphere or is equal to a closed hemisphere. The first alternative is not possible since the solution is quasi-ancient. We conclude that $\del \hat{M}_{-T_S}=M_{-T_S}$ is an equator of $\S^{n+1}.$
}


\subsection{Aleksandrov Reflection}

In this section, we use Theorem \ref{thm8} to classify convex, embedded ancient solution of the mean curvature flow on \(\S^{n+1}\). The proof uses the Aleksandrov reflection as in \cite{bryanlouie}.

We begin with some preliminaries of the Aleksandrov reflection on \(\S^{n+1}\). First, we will work relative to the limiting equator obtained in Theorem \ref{thm8}. Let \(E\) be an equator that bounds the \emph{open} hemispheres \(H^{\pm}\) with centers \(\pm \basepoint\), and let \(\vertvec = \overrightarrow{\origin\basepoint}\) be the unit vector in \(\R^{n+2}\) that points from the origin \(\origin\) to \(\basepoint\). Let \(\radialdistance(x) = d_{\S^{n+1}} (\basepoint, x)\) denote the spherical distance from \(\basepoint\) to \(x \in \S^{n+1}\). The radial projection onto \(E\) is the map \(x \in \S^{n+1} \mapsto \radialprojection(x) \in E\), where \(\radialprojection\) is the nearest point on \(E\) to \(x\). If \(x \ne \pm \basepoint\), then \(\radialprojection(x)\) is a single point. If \(x = \pm \basepoint\), then \(\radialprojection(x) = E\). In any event, given \(y \in \radialprojection(x)\), there is a unique length minimizing geodesic joining \(x\) to \(y\) and this geodesic must pass through \(\pm \basepoint\).

It is convenient to make use of the ambient \(\R^{n+2}\) and define the height function \(\height(x) = \ip{x}{\vertvec}\). The radial distance is related to the height function via
\[
\height(x) = \cos(\radialdistance(x))
\]
which is monotonically decreasing in \(\radialdistance\).

Now for the Aleksandrov reflection, let \(\reflectionvector \in \R^{n+2}\) be any unit vector that \(\ip{\reflectionvector}{\vertvec} < 0\). Let \(\reflectionplane = \reflectionvector^{\perp}\) be the hyperplane through the origin with the normal vector \(\reflectionvector\). Let \(\reflectionhalfspace^{\pm} = \{\pm \ip{x}{\reflectionvector} > 0\}\) denote the halfspaces with the boundary \(\reflectionplane\). For any subset \(S \subset \S^{n+1}\), write \(\reflectionset{S}^{\pm} = S \intersect \reflectionhalfspace^{\pm}\). Lastly, let \(\delta > 0\) denote the angle \(\reflectionvector\) makes with \(E\); therefore, \(\sin \delta = \ip{\reflectionvector}{-\vertvec}\).

\begin{definition}
The Aleksandrov reflection across \(\reflectionplane\) is the map defined by
\[
\reflectionmap: x \in \R^{n+1} \mapsto x - 2\ip{x}{\reflectionvector} \reflectionvector.
\]
\end{definition}

This map is an (orientation reversing) isometry of \(\R^{n+2}\) fixing \(\reflectionplane\) and in particular fixing the origin. Therefore, it induces an isometry of \(\S^{n+1}\). For \(x \in E\), we have \(\ip{x}{\vertvec} = 0\) and
\[
\height(\reflectionmap(x)) = \ip{\vertvec}{x - 2 \ip{x}{\reflectionvector} \reflectionvector} = 2 \sin\delta \ip{x}{\reflectionvector}.
\]
In the case \(x \in E^+\), \(\height(\reflectionmap(x)) > 0\), and in the case \(x \in E \intersect \reflectionplane\), \(\height(\reflectionmap(x)) = 0\).

In geodesic polar coordinates, \((\radialdistance, \sigma) \in (0, \pi) \times E \simeq \S^{n+1}\backslash \{\pm \basepoint\}\), a smooth, closed hypersurface \(M \subset \reflectionhalfspace^+\) that bounds a region in \(\reflectionhalfspace\) is a smooth graph \((f(\sigma), \sigma)\) over \(E\) if and only if its (outer) normal \(\nor\) satisfies \(\ip{\nor}{\vertvec} < 0\) (e.q. \(M\) has no vertical tangents). In particular, for all \(\epsilon>0\) there is a \(\xi>0\) such that if \(M\) is a graph with \(\ip{\nor}{\vertvec} < -\xi\) and \(N\) is \(\epsilon\)-close to \(M\) in \(C^{\infty}\), then \(N\) is a graph over \(E\).

\begin{theorem}
Let \(M_t\) be a convex, embedded ancient solution of the mean curvature flow on \(\S^{n+1}\). Then \(M_t\) is a family of shrinking geodesic spheres emanating from an equator at \(t=-\infty\).
\end{theorem}

\begin{proof}
Let \(E = M_{-\infty} \simeq \S^n\) be the limiting equator \(\lim\limits_{t\to-\infty} M_t\).

Since \(M_t\) smoothly converges to \(E\) as \(t\to-\infty\), we may write \(M_t\) as the graph of a smooth positive function over \(E\) in the geodesic polar coordinates, \(M_t = \{(f_t(\sigma), \sigma) \in (0,\pi) \times E\)\}. We have \(f_t \to \pi/2\) smoothly and uniformly in \(C^{\infty}\) as \(t \to -\infty\). Moreover, for \(\delta \in (0,\pi/4)\), \(\reflectionmap(E) = \{(g_{-\infty}(\sigma), \sigma)\)\} is a graph over \(E\).

We want to send \(\delta \to 0\) eventually. Let us fix a \(\delta_0 \in (0,\pi/4)\) to give us a little room because \(\reflectionmap(E)\) is not a graph when \(\delta = \pi/4\) (it is an equator perpendicular to \(E\)). Explicitly, given any \(\delta \in (0,\delta_0)\), if \(f_t\) is sufficiently close (independently of \(t\)) in the \(C^{\infty}\) topology to the equator, then \(\reflectionmap(M_t)\) may written as graph over the equator. Consequently, since \(f_t \to \pi/2\) smoothly, there is a \(T_{\delta} < 0\) such that for every \(t \leq T_{\delta}\), \(\reflectionmap(M_t) =\{ (g_t(\sigma), \sigma)\)\} is a graph.

As noted above, \(\height(\reflectionmap(x)) > 0\) on \(E^+\) and \(\height(x) = \cos(\radialdistance(x))\). Thus, continuity implies that for \(\epsilon > 0\) there exits an \(\eta>0\) such that \(\radialdistance(\reflectionmap(x)) < \pi/2 - \epsilon\) provided \(x \in E_{\eta} = \{x \in E: d(x, E \intersect \reflectionplane) > \eta\}\). By making \(T_{\delta}\) smaller, since \(M_t \to_{C^{\infty}} E\), we can arrange that \(d(M_t, E) < \epsilon/2\) for all \(t < T_{\delta}\); that is, \(\radialdistance (x) > \pi/2 - \epsilon/2\) for all \(x \in M_t\). Now for \(x \in M_t^+ \intersect \radialprojection^{-1} E_{\eta}\), since \(\reflectionmap\) is an isometry, we have \(d(\reflectionmap(x), \reflectionmap(\radialprojection(x))) < \epsilon/2\); therefore, \(\radialdistance(\reflectionmap(x)) < \pi/2 - \epsilon/2\). Consequently, away from the strip \(\{x \in E: d(x, E \intersect \reflectionplane) \leq \eta\}\), we have \(\radialdistance(\reflectionmap(\reflectionset{(M_t)}^+)) < \pi/2 - \epsilon/2\) and \(\radialdistance(M_t^-) > \pi/2 - \epsilon/2\). That is, away from the strip, we have \(\reflectionmap(\reflectionset{(M_t)}^+) > \reflectionset{(M_t)}^-\).

Now let \(C\subset E\) be any great circle and consider \(f_t|_C\). Using the Backwards Approximate Symmetry Lemma \cite[Lemma 5.1]{bryanlouie} (which applies whenever \(f_t\) converges smoothly to \(C\)), we find that (possibly by decreasing \(T_{\delta}\)) over \(C\)
\[
\reflectionmap(\reflectionset{(M_t)}^+) \geq \reflectionset{(M_t)}^-
\]
on the strip \(\{x \in E: d(x, E \intersect \reflectionplane) \leq \eta\}\). Here, \(T_{\delta}\) depends only on \(\|f_t - \pi/2\|_{C^{\infty}}\) and \(\delta\); therefore, \(T_{\delta} > - \infty\) is independent of \(C\). Thus we find that
\begin{equation}
\label{eq:backwards_approximate_symmetry}
\reflectionmap(\reflectionset{(M_t)}^+) \geq \reflectionset{(M_t)}^-
\end{equation}
everywhere for all \(t \in (-\infty, T_{\delta})\).

Let us now define \(T_{\delta}\) so that \((-\infty, T_{\delta})\) is the largest interval on which the relation \eqref{eq:backwards_approximate_symmetry} holds. Also define \(T = \inf\limits_{\delta \in (0,\delta_0)} T_{\delta}\). We want to show that \(T > -\infty\), and hence that the relation \eqref{eq:backwards_approximate_symmetry} holds on the non-empty, open interval \((-\infty, T)\). To show $T>-\infty$, we apply the maximum principle as in \cite[Lemma 5.2]{bryanlouie}: We know that \(\reflectionset{(M_t)}^-\) and \(\reflectionmap(\reflectionset{(M_t)}^+)\) lie in the interior of \(\reflectionhalfspace^-\) with the common boundary lying in \(\reflectionplane \intersect \S^{n+1}\). We also know that in a neighborhood of \(\reflectionplane \intersect \S^{n+1}\), the hypersurfaces \(\reflectionset{M_t}^-\) and \(\reflectionmap(\reflectionset{(M_t)}^+)\) are disjoint for \(t\) sufficiently negative (depending on \(\delta\)) because \(\reflectionset{M_t}^-\) is \(C^1\)-close to the equator, while \(\reflectionmap(\reflectionset{(M_t)}^+)\) is \(C^1\)-close to the reflected equator. The strong maximum principle, using a parabolic version of the Hopf boundary point lemma (similar to \cite[Theorem 2.2]{MR1483984}), ensures that the  relation \(\reflectionmap(\reflectionset{(M_t)}^+) \geq \reflectionset{(M_t)}^-\) is preserved along the flow as long as both \(\reflectionset{M_t}^-\) and \(\reflectionmap(\reflectionset{(M_t)}^+)\) are non-empty and intersect \(\reflectionplane\) transversely, and these latter conditions are true on an open interval \((-\infty, S)\) with \(S>-\infty\) independent of \(\delta\) (see \cite[Lemma 5.2]{bryanlouie}). In summary, since the relation \eqref{eq:backwards_approximate_symmetry} is true on \((-\infty, \min\{T_{\delta},S\})\), the maximum principle applied in the time interval $[\min\{T_{\delta},S\}/2,S)$ ensures that
\begin{equation*}
\label{eq:longtime_approximate_symmetry}
\reflectionmap(\reflectionset{(M_t)}^+) \geq \reflectionset{(M_t)}^-
\end{equation*}
for all \(t \in (-\infty, S)\) and any \(\delta \in (0,\delta_0)\). Hence $T\ge S.$

To complete the proof, we use \cite[Proposition 5.3]{bryanlouie} (which applies in any dimension) to conclude that \(M_t\) is a geodesic sphere for all \(t \in (-\infty, T)\) and thus for all negative times by the uniqueness of solutions.
\end{proof}


\bibliographystyle{amsplain}
\bibliography{Bibliography.bib}


\end{document}
