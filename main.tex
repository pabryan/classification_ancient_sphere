\documentclass{amsart}

\input{StandardPaper2.tex}
%\usepackage[left=1in,right=1in,top=1in,bottom=1in]{geometry}
\begin{document}

\title[Ancient solutions to curvature flows in the sphere]
 {On the classification of ancient solutions to curvature flows on the sphere}

\curraddr{}
\email{}
\date{\today}

\dedicatory{}
\subjclass[2010]{}
\keywords{}

\begin{abstract}
We consider the evolution of hypersurfaces on the unit sphere $\mathbb{S}^{n+1}$ by homogeneous functions of principal curvatures. We introduce the notion of `quasi-ancient' solutions, which plays a similar role as ancient solutions for flows that do not admit non-trivial ancient solutions. The techniques presented here allow us to prove that any convex, quasi-ancient solution of a curvature flow which satisfies a uniform bound on the second fundamental form backwards in time must be a family of shrinking geodesic spheres. The main tools are rigidity result in the sphere and an Aleksandrov reflection argument.
\end{abstract}

\maketitle

\section{Introduction}
\label{sec:intro}

We consider the evolution of a hypersurface $M^n$ by
\eq{\label{eq:CurvFlow}
\partial_tx=-F(\mathcal{W})\nu,~ x:M^n\times[0,T)\to M_K,
}
where \(M_K\) is the simply connected space form of constant sectional curvature \(K>0\) (just say the sphere?), \(\mathcal{W}\) is the Weingarten map and \(F : \Gamma(T^\ast M \otimes TM) \to C^{\infty}(M)\) is a smooth map such that for any \(A, B \in \Gamma(T^\ast M \otimes TM)\), we have pointwise relation,
\[
A < B \Rightarrow F(A) < F(B).
\].
This latter condition ensures the flow is strictly parabolic. In particular, it also ensures that if \(M_t = x_t(M)\) is strictly convex, then \(F(\mathcal{W} > 0\). Since \(\mathcal{W}\) is symmetric, it is diagonalisable and so \(F\) induces a symmetric function on the eigenvalues. Conversely, any such function induces an \(F\). Expressed in terms of \(\kappa_1,\cdots,\kappa_n)\), the eigenvalues of \(\mathcal{W}\) (principal curvatures), the parabolicity condition translates to 
\[
\frac{\partial F}{\partial \kappa_i} > 0.
\]

We consider /quasi-ancient/, strictly convex solutions of \eqref{eq:CurvFlow}. Here quasi-ancient refers to solutions existing on the same time interval as the maximal flow of geodesic spheres. For many common flows, such as the Mean Curvature Flow (\(F(\mathca{W}) = \text{Trace}(\mathcal{W}) = H\), the maximal flow of geodesic spheres is the infinite time interval \((-\infty, 0)\). However for flows \(F = H^p\) with \(p \in (0,1)\), the maximal time interval is in fact finite. See \cref{sec:quasi_ancient}.

We prove the following theorem,
\begin{theorem}
Let \(M_t\) be a /quasi-ancient/, strictly convex solution of \eqref{eq:CurvFlow} with bounded mean curvature. Them \(M_t\) is a flow of shrinking geodesic spheres.
\end{theorem}

In certain situations, such as concave speeds with speed comparable to \(H\), or one-homogeneous convex speeds, the theorem is actually quite easy to prove and we offer a short proof in this case in \cref{sec:convex_concave_homogeneous}. This should be compared, for instance with the result in \cite[Theorem]{huiskensinestrari} where a short proof in the case of the Mean Curvature Flow in the sphere may be found. The main idea is to find the correct quantity to which the maximum principle may be applied. In the referred to paper, it is the pinching quantity, \((|A|^2 - n H^2)/H^2\) used so often for the Mean Curvature Flow. Here, we use \(\mathcal{W}/F\) for concave, mean curvature comparable speeds and the non-collapsing quantity \(k\) for one-homogeneous, convex speeds used in \cite{andrewsetal}.

The main thrust of the theorem, is that we have very strong rigidity in the sphere, and are able to prove the result in great generality - /all we require is parabolicity of the flow/. At this level of generality, it is quite difficult to obtain suitable estimates, in particular Evans-Krylov higher regularity estimates are not known for arbitrary non-linear equations and hence less direct PDE methods are required. We obtain the theorem by first using the rigidity results of \cite{matowskischeuer} to argue that quasi-ancient solutions with bounded \(H\) must limit to an equator backwards in time. Then the Aleksandrov reflection technique developed in \cite{bryanlouie,bryanivaki} applies to show the symmetry is preserved under the flow, completing the classification. Let us remark that one may seek other method of classification, using more PDE theoretic techniques, such as semi-group methods and centre manifold analysis to deduce linear stability near the equator. These methods are quite broadly applicable, but suffer the draw back that they only apply (without modification) to speeds \(F\) that are non-singular and non-degenerate on the equator. The methods we describe here, apply even in the case of /singular or degenerate speeds on the equator/. Very simple examples of such speeds are \(H^p\) which is singular for \(0 < p < 1\) and degenerate for \(p > 1\) whenever \(H = 0\), in particular along the equator. We find that the geometric methods are quite appealing and of interest in their own right, even in cases where PDE methods are applicable, these methods offering a powerful, complementary alternative to direct PDE techniques.

It's also worth comparing the techniques and results here to the Euclidean case. The latter does not exhibit such strong rigidity and other ancient, convex solutions may occur. Indeed, for the Curve Shortening Flow in the plane, \cite{hamdaskses} have classified all such solutions as either shrinking circles or the Angenent oval (also known as the paper clip solution) \cite{ang}. In higher dimensions, \cite{huiskensinestrari} and \cite{kleinerhaslhoffer} have found a set of equivalent conditions characterising when an ancient, convex mean curvature flow is by shrinking spheres, and one would generally expect a greater variety of convex ancient solutions to exist.

This paper is laid out as follows: preliminaries, quasi-ancient, convex/convex speeds, rigidity, reflection.

\section{Preliminaries}
\label{sec:prelim}

We will need some notation for derivatives of the speed \(\varphi\). Let us write
\[
\varphi^{i}_{j} = \frac{\partial \varphi}{\partial h^{j}_{i}}
\]
for the first partial derivatives of \(\varphi\). As a function of the metric and second fundamental form,
\[
\varphi(g, h) = \varphi(g^{ik} h_{kj}),
\]
we write
\[
\varphi^{ij} = \frac{\partial\varphi}{\partial h_{ij}}, \quad \varphi^{ij,kl} = \fr{\partial^2\varphi}{\partial h_{kl} \partial h_{ij}}.
\]
Let us also define the operator
\[
\Box = \varphi^{ij} \nabla^2_{ij}.
\]
\begin{lemma} \label{lem: basi ev}
The following evolution equations hold:
\eq{
 \label{eq:delt_weingarten_box} 
\partial_t h_i^j &= \Box h_i^j + \varphi^{kl} (h^2)_{kl} h_i^j - (\varphi^{kl}h_{kl} - \varphi) (h^2)_i^j + \varphi^{kl,rs}\nabla_i h_{kl}\nabla^j h_{rs} \\
& \quad + K \{(\varphi + \varphi^{kl}h_{kl}) \delta_i^j - \p^{kl}g_{kl} h_i^j\},
}
\eq{\label{eq:delt_speed} \partial_t \varphi = \Box \varphi + \varphi\varphi^{ij}(h^2)_{ij} + K \varphi\varphi^{ij}g_{ij}.}
\end{lemma}

We will have occasion to work with graphs \(u\) over an equator, \(E \subset \mathbb{S}^{n+1}\).

A straightforward computation yields the following representation of the Weingarten map in terms of the function $u,$ namely
\eq{\label{graph h}h^i_j=\fr{\vt'}{v\vt}\d^i_j+\fr{\vt'}{v^3\vt^3}\nabla^iu\nabla_ju-\fr{\~g^{ik}}{v\vt^2}\nabla^2_{kj}u,}
where \(\nabla\) denotes $\~g^{ij}$ is the inverse of $\~g_{ij}=\vt^{-2}g_{ij},$ $\vt(r)=\sin r$ and covariant derivatives as well as index raising is performed with respect to $\s_{ij},$ the covariant derivative on the equator, \(E \simeq \S^n\). Compare for example \cite[(3.82)]{Scheuer:05/2015}.

A one-parameter family of graphs satisfying the flow \eqref{eq:CurvFlow}, thus is a solution of
\begin{equation}
\label{eq:GraphCurvFlow}
\fr{\del}{\del t}u=F\br{\fr{\vt'}{v\vt}\d^{i}_{j}+\fr{\vt'}{v^{3}\vt^{3}}\nabla^{i}u\nabla_{j}u+\fr{\~g^{ik}}{v\vt^{2}}\nabla^{2}_{kj}u}=\Phi(x,u,\nabla u,\nabla^{2}u).
\end{equation}

The paraboicity assumption is that \(dF > 0\) and writing,
\[
\Phi = \Phi(x, r, p, z)
\]
we find that linearising around \(u_0\) symbol is,
\[
\sigma(u_0) \cdot X = d\Phi_z \cdot X = dF \cdot \fr{1}{v(u_0) \vt^{2}(u_0)} \~g^{-1}(u_0) \cdot X
\]
for \(X \in TC^{\infty}(E, \mathbb{R}) \simeq C^{\infty}(E, \mathbb{R})\). This is uniformly elliptic wherever \(\~g^{-1}(u_0)\) is, and wherever, \(v(u_0), \vt(u_0)\) are bounded below. In particular, this is true if the hypersurface represented by the graph \(u_t\) is strictly convex with second fundamental form, \(h(u_t)\) bounded below.

\section{Ancient and quasi-ancient Solutions}
\label{sec:quasi_ancient}

We consider the spherical ambient space, $K=1,$ without further mention, and we are interested in solutions with maximal possible lifetime. To understand this maximal time, we define $T_S$ to be the lifespan of \it{the} convex spherical solution of \eqref{eq:CurvFlow}. By the convex spherical solution we mean a family of geodesic spheres shrinking under the flow \eqref{eq:CurvFlow} collapsing to a point at time $t=0$ and existing on the maximal interval \((-T_S, 0)\). For 1-homogeneous $\p$, \(T_S = \infty\), but for $\alpha$-homogeneous $\p$ with $\alpha<1$, \(T_S\) is finite.
\begin{lemma}
 Consider \eqref{eq:CurvFlow} with speed \(\p = f^{\alpha}\) for \(\alpha \in (0,1)\). Then a flow of strictly convex geodesic spheres has a finite lifespan, i.e., let $S_r(p)$ be a geodesic sphere in $\S^{n+1}$ around $p\in \S^{n+1}$. Then the flow exists only for a finite time interval \((-T_S,0)\) with \(0 < T_S < \infty\), collapsing to a point at \(t=0\) and converging to an equator at \(t=T_S\).
\end{lemma}
\begin{proof}
We may assume that $f(1,\cdots,1)=n.$
Since $f$ is constant on a geodesic sphere, the evolution equation (\ref{lem: basi ev}) for a flow of geodesic spheres yields
\eq{\fr{d}{dt}f\geq nf^{\a}.}
Since the right hand side remains strictly positive under this ODE, we obtain finite lifespan forward in time.
Convexity and integration over some interval $(a,b)$ yield
\eq{0\leq f^{1-\a}(a)\leq f^{1-\a}(b)-n(1-\a)(b-a).}
Allowing $a\ra-\8$ gives the finite existence backwards in time.
\end{proof}
\begin{lemma}
Let $x$ be a convex solution of \eqref{eq:CurvFlow}, defined on the open interval $(-T,0),$ where $0$ is the collapsing time, then $T\leq T_S.$
\end{lemma}

\begin{proof}
Suppose $T>T_S+\e$ for some $\e>0$. Since $M=M_{-T_S-\fr{\e}{2}}$ bounds a convex body $\hat{M}$, it is strictly contained in an open hemisphere due to the classical paper \cite{CarmoWarner:/1970}. Then there exists a geodesic sphere $S$ with $\hat{M}\sub\hat{S}.$ By the avoidance principle, the flow with initial hypersurface $M$ collapses before the spherical flow and thus contradicting $T>T_S$.
\end{proof}
Due to this lemma, the following definition is reasonable.
\begin{definition}
A convex solution of \eqref{eq:CurvFlow} defined on maximal interval $(-T,0)$ is called \it{quasi-ancient}, if $T=T_S$.
\end{definition}
The term ancient is reserved for the situation when \(T_S=\infty\), and by the definition, ancient solutions are also quasi-ancient.

\section{Convex, Concave and Homogeneous Speeds}
\label{sec:convex_concave_homogeneous}

\begin{proposition}
Let $f$ be concave and 
$\fr{H}{f}\leq c,$
or let $f$ be convex. Then any strictly convex ancient solution of the flow \eqref{eq:CurvFlow} with speed $\p(f)=f$ is a family of contracting geodesic spheres.
\end{proposition}
\pf{
We first consider the case that $f$ is a concave function. The tensor
$w^j_i=\fr{h^j_i}{f}$
satisfies the evolution equation
\eq{\del_tw^j_i&=\Box w^j_i-2Kf^{kl}g_{kl}w^j_i+2K\d^j_i\\
 &\hp{=}+f^{kl,rs}\nabla_ih_{kl}\nabla^jh_{rs}+2f^{kl}\nabla_kh^j_i\nabla_l\br{\fr{1}{f}}.}
Hence, due to the concavity of $f$ we have for fixed index $i:$
$$\del_t\br{w^i_i-\fr{1}{n}}\leq\Box\br{w^i_i-\fr{1}{n}}-2nK\br{w^i_i-\fr{1}{n}}+2f^{kl}\nabla_kh^i_i\nabla_l\br{\fr{1}{f}}.$$
At a spatial maximum of $w^i_i-\tfrac{1}{n}$ we have
$$0=\nabla_k \br{\fr{h^i_i}{f}}=\fr{\nabla_k h^i_i}{f}+h^i_i\nabla_k\br{\fr{1}{f}}.$$
and hence by the maximum principle we obtain the bound
$$w^i_i-\fr 1n\leq c_0e^{-2nKt},$$
where $c_0$ is the upper bound for the function on the left-hand side at $t=0.$
Starting the flow at an arbitrary $s<0$ we find
$$\fr{\k_n}{f}\leq \fr 1n+c_se^{-2nK(t-s)}.$$
If $K=1,$ letting $s\ra-\8,$ we obtain at every time $t<0$ that 
$$\k_n\leq \fr{f}{n}.$$
This is only possible for totally umbilical hypersurfaces.   

Now suppose that $f$ is a convex function and $f(1,\cdots,1)=n$. The non-collapsing result of \cite[Theorem 1.1]{andrews2015Non-collapsing} states any ancient solution of the flow with a 1-homogeneous convex speed $f$ satisfies
\[\frac{\kappa_{\min}}{f}(x,t)\geq \frac{1}{n}+C(s)e^{-2n(t-s)},\]
where $|C(s)|\leq \frac{1}{n}+\frac{\kappa_{\min}}{f}(x,s)\leq \frac{1}{n}+\frac{\kappa_{min}}{\frac{f(1,\cdots,1)}{n}H}(x,s)\leq \frac{1}{n}+\frac{1}{f(1,\cdots,1)}.$
Therefore, allowing $s\to-\infty$ yields
\[\frac{\kappa_{\min}}{f}(x,t)\geq \frac{1}{n}.\]
This in turn implies that
\[\kappa_{\min}(x,t)\geq \frac{f}{n}(x,t)\geq \frac{f(1,\cdots,1)}{n}\frac{H}{n}(x,t)\geq \frac{f(1,\cdots,1)}{n} \kappa_{\min}(x,t).\]
On the other hand, since $f(1,\cdots,1)= n,$ we get
\[\kappa_{\min}(x,t)\geq \frac{f}{n}(x,t)\geq \frac{H}{n}(x,t)\geq \kappa_{\min}(x,t).\]
That is, $H\equiv n\kappa_{\min}.$
}

\section{Rigidity and Backwards Limit}

The aim of this section is to prove that for a quasi-ancient solution of \eqref{eq:CurvFlow} the backwards limit of the flow hypersurfaces \it{with bounded mean curvature}, $M_t$ is an equator for $t\ra -T_S$. We will use the method of \cite{MakowskiScheuer:/2013} to achieve this.
For 1-homogeneous, convex speeds, Proposition \ref{cor:boundedH} gives a bound on mean curvature for ancient solutions. For quasi-ancient solutions, the Harnack inequality does not, in general, give such a bound, since we cannot send \(t \to -\infty\) whenever \(T_S\) is finite. One can envisage backwards limits as convex polyhedra and hence with unbounded \(H\), but it is not clear that these can arise as backwards limits of quasi-ancient solutions. Thus, at this stage, we must make the additional assumption that \(H\) is bounded for quasi-ancient solutions.

\begin{proposition}
\label{cor:boundedH}
Suppose $f$ is a convex curvature function. Then any convex ancient solution of the contracting flow with speed $\p(f)=f$ satisfies
\[\partial_t \p-b^{ij}\nabla_i\p\nabla_j\p\geq 0.\]
In particular, for all $t\le -1$ we have
$H(\cdot,t)\leq c.$
Here $c<\infty$ depends only on $M_{-1}.$
\end{proposition}
\begin{proof}
For any $t>s$, the  Harnack estimate of \cite[Theorem 1]{bryan2015harnack} implies that
$$\partial_t \p-b^{ij}\nabla_i\p\nabla_j\p+\frac{n}{2}\frac{\p}{t-s}>0.$$
Allowing $s\to-\infty$ proves the first claim. For the second claim, observe that for any 1-homogeneous convex $f$ we have \[f\ge \frac{f(1,\cdots,1)}{n}H,\]
see \cite[Chapter 2]{Gerhardt:/2006}. Therefore, ancient solutions satisfy
\[H(\cdot,t)\leq \frac{n}{f(1,\cdots,1)}\p(\cdot,t)\leq \frac{n}{f(1,\cdots,1)}\p(\cdot,0). \]
\end{proof}
\begin{lemma}\label{ISC}
Let $x$ be a quasi-ancient solution of \eqref{eq:CurvFlow}. Then there holds:
\begin{enumerate}
  \item For all $t_0<0$ there exists a uniform radius $R>0,$ such that the enclosed convex bodies $\hat{M}_t,$ $-T_S<t\leq t_0,$ of the flow hypersurfaces $M_t$ satisfy a uniform interior sphere condition with radius $R.$
  \item For every $y_0\in\mrm{int}~\hat{M}_{t_0}$ the hypersurfaces $M_t,$ $-T_S<t\leq t_0$ can be written as a graph in geodesic polar coordinates around $y_0$ and the corresponding graph functions satisfy uniform $C^2$-estimates.
\end{enumerate}
\end{lemma}
\pf{
Fix an interior point $y_0\in \mrm{int}~\hat{M}_{t_0}.$ Since for a contracting flow the enclosed convex bodies of the flow hypersurfaces are strictly decreasing, they are strictly increasing backwards in time. By \cite[Lemma~3.9]{MakowskiScheuer:/2013} there exists a closed hemisphere $\mc{H}(x_0),$ such that
\eq{\hat{M}_t\sub\mc{H}(x_0).}
In our situation all hypersurfaces $M_t,$ $-T_S<t\leq t_0,$ satisfy
\eq{B_{\e}(y_0)\sub \mrm{int}~\hat{M}_t}
and
\eq{B_{\e}(\hat{y}_0)\sub \hat{M}_t^c}
with a uniform $\e,$ where $\hat{y}_0$ denotes the antipodal point of $y_0.$
Now we prove the two claims.

(1)~Consider the stereographic projection with $\hat{y}_0$ corresponding to infinity. The image hypersurfaces are then strictly convex hypersurfaces in Euclidean space with uniformly bounded second fundamental form. Blaschke's rolling theorem, cf.~\cite{Blaschke:/1956}, gives the interior sphere condition.\\

(2)~
Write the $M_t$ as graphs in geodesic polar coordinates around $y_0,$
\eq{M_t=\{(r,x^i)\cn r=u(t,x^i)\},}
where $r$ describes the geodesic distance to $y_0.$ In these coordinates the spherical metric takes the form
\eq{d\-s^2=dr^2+\sin^{2}r\s_{ij}dx^idx^j,}
where $(\s_{ij})$ is the round metric of $\S^n.$

Hence on the set in which the $M_t$ range, the metrics $\-g_{ij}=\sin^2r\s_{ij}$ and $\s_{ij}$ are equivalent.
Due to \cite[Thm.~2.7.10]{Gerhardt:/2006} for all convex hypersurfaces $M_t$ the quantity
\eq{v^2=1+\-g^{ij}\nabla_iu\nabla_ju}
is uniformly bounded by a constant which only depends on $\e.$
Hence by the equivalence of norms the $M_t$ are uniformly $C^1$-bounded in the sense that the corresponding functions $u(t,\cdot)$ are uniformly $C^1(\S^n)$-bounded. Recalling equation, \eqref{eq:graph h}, due to the curvature estimates we obtain uniform $C^2(\S^n)$-estimates for \(u\).}
\begin{corollary}\label{Backlimit}
Let $x$ be a quasi-ancient solution of \eqref{eq:CurvFlow}. Then there exists a unique backwards limiting hypersurface $M_{-T_S}$ and the flow hypersurfaces $M_t$ converge to $M_{-T_S}$ in $C^{1,\b},$ $0<\b<1,$ in the sense that for a common graph representation as in Lemma \ref{ISC} there holds
\eq{u(t,\cdot)\ra u(-T_S,\cdot)}
in the norm of $C^{1,\b}(\S^n).$
\end{corollary}
\pf{
Due to the point-wise monotonicity of $u(t,\cdot)$ backwards in time, we obtain a point-wise limit. The $C^{1,\b}$-convergence follows from compactness.
}
\begin{theorem}
\label{thm:backwardslimit}
The hypersurface $M_{-T_S}$, defined in Corollary \ref{Backlimit}, is an equator.
\end{theorem}
\pf{
Since the convex bodies $\hat{M_t}$ are increasing backwards in time and due to the uniform convergence of $M_t$ to $M_{-T_S},$ the set
\eq{\hat{M}_{-T_S}:=\overline{\bigcup_{t<0}\hat{M}_t}}
is a compact body with
\eq{\del \hat{M}_{-T_S}=M_{-T_S}.}
Since $\mrm{int}(\hat{M}_{-T_S})$ is a strictly convex set, it is especially weakly convex in a hemisphere in the sense of \cite[Def.~3.2]{MakowskiScheuer:/2013}. Thus $\hat{M}_{-T_S}$ is a weakly convex body in a hemisphere. The proof of \cite[Lemma~6.1]{MakowskiScheuer:/2013} can literally be applied to show that $\hat{M}_{-T_S}$ satisfies a uniform interior sphere condition as well.
We can apply \cite[Thmeorem 1.1]{MakowskiScheuer:/2013} and obtain that $\hat{M}_{-T_S}$ is either strictly contained in an open hemisphere or is equal to a closed hemisphere. The first alternative is not possible since the solution is quasi-ancient. We conclude that $\del \hat{M}_{-T_S}=M_{-T_S}$ is an equator of $\S^{n+1}.$
}

\section{Aleksandrov Reflection and Classification}

In this section, we use Theorem \ref{thm:backwardslimit} to classify convex, embedded, (quasi-)ancient solutions of contracting curvature flows on \(\S^{n+1}\) as either equators or shrinking geodesic spheres. The proof uses Aleksandrov reflection as in \cite{bryanlouie,2015arXiv150802821B}. Here we give a very general version with minimal assumptions on the flow: all we require is that the flow limits to an equator at $-T_S$ and that the maximum principle holds forward in time.

We begin with some preliminaries of the Aleksandrov reflection on \(\S^{n+1}\). First, we will work relative to the limiting equator obtained in Theorem \ref{thm:backwardslimit}, denoted by $\equator = M_{-T_S}$. The equator \(\equator\) determines two \emph{open} hemispheres \(H^{\pm}\) with centers \(\pm \basepoint\) and we assume the flow is contained in the upper hemisphere, $M_t \subset H^+$. It's convenient to make use of the ambient Euclidean space, \(\R^{n+2}\) with \(\S^{n+1} \subset \R^{n+2}\) via the standard embedding. Let \(\vertvec = \overrightarrow{\origin\basepoint}\) be the unit vector in \(\R^{n+2}\) that points from the origin \(\origin\) to \(\basepoint\); the ``vertical direction''. 

To define the Aleksandrov reflection, let \(\reflectionvector \in \R^{n+2}\) be any unit vector satisfying \(\ip{\reflectionvector}{\vertvec} \leq 0\). Let \(\reflectionplane = \reflectionvector^{\perp}\) be the hyperplane through the origin orthogonal to \(\reflectionvector\) and let \(\reflectionhalfspace^{\pm} = \{\pm \ip{x}{\reflectionvector} > 0\}\) denote the open half-spaces with boundary \(\reflectionplane\). For any subset \(S \subset \S^{n+1}\), write \(\reflectionset{S}^{\pm} = S \intersect \reflectionhalfspace^{\pm}\). Lastly, let \(\reflectionangle \geq 0\) denote the angle \(\reflectionvector\) makes with \(\equator\) so that \(\sin \reflectionangle = \ip{\reflectionvector}{-\vertvec}\).
\begin{definition}
The Aleksandrov reflection across \(\reflectionplane\) is the map defined by
\[
\reflectionmap: x \in \R^{n+2} \mapsto x - 2\ip{x}{\reflectionvector} \reflectionvector.
\]
\end{definition}
The reader may find it useful to refer to \Cref{fig:reflection} for the arguments in this section.
\begin{figure}[htb]
\centering
\includegraphics[width=.9\linewidth]{./reflection.pdf}
\caption{Reflection in the $(\vertvec, \reflectionvector)$-plane showing the reflected equator, \(M_T\), \(\reflectionmap(M_t)\) and some geodesics through the north pole (dotted lines).}
\label{fig:reflection}
\end{figure}
This map is an idempotent, (orientation reversing) isometry of \(\R^{n+2}\) fixing \(\reflectionplane\) and in particular fixing the origin. Therefore, it induces an idempotent isometry of \(\S^{n+1}\). Our aim is to show that for any \(\reflectionvector\) with \(\reflectionangle = 0\), the flow \(M_t\) is invariant under \(\reflectionmap\). This will complete the classification since the invariance implies that at each time \(t\), \(M_t\) is contained in a hyperplane orthogonal to \(\vertvec\) and hence must be a geodesic sphere. To achieve this goal, we first work with ``perturbed reflections'' satisfying the condition \(\reflectionangle > 0\) to obtain estimates and then finally we send \(\reflectionangle \to 0\). 

The relevant estimates are comparisons of the spherical distance from the reflection \(\reflectionmap(M_t)\) to the equator with the spherical distance of the original hypersurface \(M_t\) to the equator. With this comparison in mind, let \(\radialdistance(x) = d_{\S^{n+1}} (\equator, x) \subseteq [-\pi/2,\pi/2]\) denote the signed, spherical distance from \(\equator\) to \(x \in \S^{n+1}\) with \(H^{\pm} = \{\pm \radialdistance > 0\}\). The radial projection onto \(\equator\) is the map \(x \in \S^{n+1} \mapsto \radialprojection(x) \in \equator\), where \(\radialprojection\) is the nearest point on \(\equator\) to \(x\). If \(x \ne \pm \basepoint\), then \(\radialprojection(x)\) is a single point. If \(x = \pm \basepoint\), then \(\radialprojection(x) = \equator\). In any event, given \(y \in \radialprojection(x)\), there is a unique length minimizing geodesic joining \(x\) to \(y\) and this geodesic must pass through \(\pm \basepoint\). For graphs over the equator \(\{r = f(\sigma) : \sigma \in \equator\}\), each such geodesic intersects the graph in exactly one point.

The height function is \(\height(x) = \ip{x}{\vertvec}\) and is related to the radial distance via
\[
\height(x) = \sin(\radialdistance(x))
\]
which is monotonically increasing in \(\radialdistance\). This monotone relation allows us to work with the height function \(h\) in the ambient Euclidean space \(\R^{n+2}\) rather than the spherical distance \(\radialdistance\) resulting in much simpler calculations.

For \(x \in \equator\), we have \(\ip{x}{\vertvec} = 0\) and
\begin{equation}
\label{eq:equatorheight}
\height(\reflectionmap(x)) = \ip{\vertvec}{x - 2 \ip{x}{\reflectionvector} \reflectionvector} = 2 \sin\reflectionangle \ip{x}{\reflectionvector}.
\end{equation}
In the case \(x \in \reflectionset{\equator}^+\), we have \(\ip{x}{\reflectionvector} > 0\) and hence \(\height(\reflectionmap(x)) > 0\). In the case \(x \in \equator \intersect \reflectionplane\), we have \(\ip{x}{\reflectionvector} = 0\) and hence \(\height(\reflectionmap(x)) = 0\). This allows us to compare \(\reflectionset{(\reflectionmap(\equator))}^-\) on which \(\height\geq 0\) (equality occurring precisely on the boundary \(\equator \intersect \reflectionplane\)) with \(\reflectionset{\equator}^-\) on which \(\height = 0\). For any hypersurface close to \(\equator\) we will then obtain a similar relation. The major difficulty occurs on the moving boundary \(M_t \intersect \reflectionplane\) on which both sets have \(\height = 0\). To make this comparison precise, we define the following relation:
\begin{definition}
For subsets \(S,T \subset \S^{n+1}\), we say \emph{\(S\) one-sided reflects above \(T\)}, written \(\reflectionmap(S^+) \geq T^-\) provided \(\radialdistance(x) \leq \radialdistance(y)\) for every \(x \in S^+\) and every \(y \in \radialprojection^{-1} (\reflectionmap(x)) \intersect T^-\). Equivalently we may require \(\height(x) \geq \height(y)\).
\end{definition}

In other words, on \(\reflectionhalfspace^-\), the minus side of \(\reflectionplane\), the reflection \(\reflectionmap(S)\) lies ``above'' \(T\).

From now on we assume that \(M_t \subset H^+\), that \(M_t \to \equator\) in \(C^1\) as \(t \to -T_S\) with a uniform \(C^2\) bound, and that \(M_t\) evolves by a parabolic equation, uniform in any region with \(h \geq C g\) for \(C>0\). These assumptions imply we may write \(M_t\) as the graph of a \(C^1\), positive function over \(\equator\) in geodesic polar coordinates: \(M_t = \{r = f_t(\sigma) \in (0,\pi/2) : \sigma \in \equator\)\} for all \(t \in (-T_S, \bar{T}_{\reflectionangle})\) with \(\bar{T}_{\reflectionangle}\) sufficiently close to \(T_S\). Moreover, for \(\reflectionangle \in (0,\pi/4)\), \(\reflectionmap(\equator) = \{r = g_{-T_S}(\sigma)\}\) is a graph over \(\equator\) with \(\ip{\nor_{\reflectionangle}}{\vertvec}\) a constant in \((0,1]\) where \(\nor_{\reflectionangle}\) is the unit normal to \(\reflectionmap(\equator)\) (which is the intersection of a hyperplane and \(\S^{n+1}\)). Since \(\reflectionmap\) is an isometry, \(\reflectionmap(M_t) \to \reflectionmap(\equator)\) in \(C^1\) as \(t \to -T_S\) and hence, increasing \(\bar{T}_{\reflectionangle}\) if necessary, we also have that \(\reflectionmap(M_t) = \{r = g_t(\sigma)\}\) is a graph for all \(t \in (-T_S, \bar{T}_{\reflectionangle})\).

We will make use of monotonicity in \(\reflectionangle\): the constant, \(\ip{\nor_{\reflectionangle}}{\vertvec}\) is monotonically increasing to \(1\) as \(\reflectionangle \to 0\). This just says that \(\nor_{\reflectionangle}\) becomes more vertical as \(\reflectionangle\) decreases and that when \(\reflectionangle = 0\), we have \(\nor_{\reflectionangle} = \vertvec\) (since \(\equator\) is invariant under \(\reflectionmap\)). To give us a little room away from \(\pi/4\) where the reflected equator is no longer a graph, fix any \(\reflectionangle_0 \in (0,\pi/4)\). Letting \(\bar{T} = \bar{T}_{\reflectionangle_0} < T_S\), from the monotonicity in \(\reflectionangle\), both \(M_t\) and \(\reflectionmap(M_t)\) are graphs for all \(t \in (-T_S, -\bar{T})\) and all \(\reflectionangle \in [0,\reflectionangle_0]\).

In the following sequence of lemmas, under the above assumptions, we prove that the hypersurface \(M_t\) one-sided reflects above itself on the interval \((-T_S, -\bar{T})\) for any \(\reflectionangle \in (0, \reflectionangle_0)\).

Let \(C\subset \equator\) be any great circle and consider \(f_t|_C\). The Backwards Approximate Symmetry Lemma \cite[Lemma 5.1]{bryanlouie} applies whenever \(f_t|_C\) converges in \(C^2\) to \(C\) to show that there is a \(T_{\reflectionangle} \in (0, T_S)\) such that
\[
\reflectionmap(\reflectionset{(M_t)}^+) \geq \reflectionset{(M_t)}^-
\]
over any \(C\) and on a neighborhood of \(\reflectionplane\) in \(\reflectionhalfspace\) for all \(t \in (-T_S, -T_{\reflectionangle})\).

Here we weaken the assumptions to only require \(C^1\) convergence with a uniform \(C^2\) bound. In general, This seems the best we can hope for as is remarked in the proof. The proof is similar to the Backwards Approximate Symmetry Lemma, but now we need to obtain better estimates on the neighbourhood. The tricky part is dealing with the moving boundary \(M_t \intersect \reflectionplane\).

\begin{lemma}
\label{lem:approximate_symmetry}
For any \(\reflectionangle \in (0,\reflectionangle_0)\) there exists a \(T_{\reflectionangle} \in (0, T_S)\) such that \((\reflectionset{\reflectionmap(M_t))}^- \geq M_t^-\) for all \(t \in (-T_S, -T_{\reflectionangle})\).
\end{lemma}
\begin{proof}
\emph{Boundary Estimates}
Let \(C\subset \equator\) be any great circle and consider \(f_t|_C\), \(g_t|_C\). We use the Taylor expansions near the two points \(\reflectionplane \intersect M_t = \{x_1(t), x_2(t)\}\) over \(C\). We work near \(x_1(t)\) (the proof being the same near \(x_2\)). For each \(t\), parametrise \(C\) with \(\theta\) such that \(\radialprojection(x_1(t))\) corresponds to \(\theta = 0\) and \(\radialprojection^{-1}\{\theta>0\} \intersect M_t\) lies in \(\reflectionhalfspace^-\). Note that since \(\reflectionmap\) preserves \(\reflectionplane\), we have \(f_t(0) = g_t(0)\) and thus the Taylor expansion with second order remainder is
\[
g_t(\theta) - f_t (\theta) = (g_t'(0) - f_t'(0)) \theta + R \theta^2
\]
where \(R\) is bounded depending only on the uniform bounds for \(f_t''\) (note \(g_t'' = f_t''\)). Since \(M_t \to \equator\) in \(C^1\), as \(t\to -T_S\) we have that \(f_t' \to 0\) and \(g_t' \to \tan(2\reflectionangle)\), the latter being the slope of the reflected equator. Thus there is a \(T_{\reflectionangle}\) such that \(g_t'(0) - f_t'(0)\) is arbitrarily close to the positive constant \(\tan(2\reflectionangle) > 0\) and hence the first order term is positive for all \(t \in (-T_S, T_{\reflectionangle})\). We can easily estimate the interval on which \(g_t(\theta) \geq f_t (\theta)\): The uniform bounds on \(f_t''\) imply that
\[
\eta = \inf_{t \in (-T_S, T_{\reflectionangle})} \frac{g_t'(0) - f_t'(0)}{|R|} > 0
\]
and thus \(g_t(\theta) \geq f_t (\theta)\) on the interval \([0, \eta)\) for any \(t \in (-T_S, T_{\reflectionangle})\). Note that previously in the Backwards Approximate Symmetry Lemma, \(C^2\) convergence to \(0\) implied that \(R\to 0\) and hence by taking \(t\) close enough to \(-T_S\) we could make \(\eta\) arbitrarily large. This of course does not apply with the relaxed assumptions of \(C^2\) bounds only, but it turns out that all we require is \(\eta > 0\). Note also that without uniform \(C^2\) bounds, \(\eta\) may in fact be \(0\) and this is where the final argument below may fail.

Now we can let \(C\) vary and observe that we can make \(g_t'(0) - f_t'(0) > 0\) and \(|R|\) bounded independently of \(C\) by the uniform \(C^1\) convergence and uniform \(C^2\) bound, thus obtaining \(\eta\) independent of \(C\). 

Now, \(\reflectionplane \intersect H^+\) projects onto \(\reflectionset{\equator}^+\), and so \(\radialprojection(x_1(t)) \in \reflectionset{\equator}^+\) and we need to extend our boundary estimate over to \(\reflectionset{\equator}^-\). By decreasing \(T_{\reflectionangle}\) if necessary, as \(M_t\intersect\reflectionplane \to \equator\intersect\reflectionplane\) we have that \(\radialprojection(M_t\intersect\reflectionplane) \subseteq \{x \in \equator : d(x, \reflectionplane) < \eta/2\}\) for all \(t \in (-T_S, T_{\reflectionplane})\). Note that \(\radialprojection(x_1(t))\) is contained is this set, and \(g_t(\theta) \geq f_t (\theta)\) holds for \(\theta\) such \(d(\radialprojection(x_1(t)), C(\theta)) < \eta\) if \(\theta\) is the arc-length parameter of \(C\). This is true for any \(C\), hence holds on \(\{x \in \reflectionset{\equator}^+ : d(x, \reflectionplane) < \eta/2\}\). In geodesic coordinates, for graphs the condition \(\reflectionmap(\reflectionset{(M_t)}^+) \geq \reflectionset{(M_t)}^-\)is equivalent to \(g_t(\theta) \geq f_t (\theta)\) hence we obtain
\begin{equation}
\label{eq:boundaryestimate}
\reflectionmap(\reflectionset{(M_t)}^+) \geq \reflectionset{(M_t)}^- \quad \text{ over } \{x \in \reflectionset{\equator}^+: d(x, \reflectionplane) < \eta/2\}
\end{equation}
and all \(t \in (-T_S, T_{\reflectionangle})\).

\emph{Interior Estimates}

As noted above, \(\height(\reflectionmap(x)) > 0\) on \(\reflectionset{\equator}^+\) and \(\radialdistance = \arcsin(\height(x))\). For any \(\mu > 0\), let
\[
\reflectionset{\equator}^{+,\mu} = \{x \in \reflectionset{\equator}^+: d(x, \equator \intersect \reflectionplane) > \mu\},
\]
and let
\[
G_{\mu} = \inf\{d(\reflectionmap(x), \equator) : x \in \reflectionset{\equator}^{+,\mu}\} > 0.
\]
Since \(M_t \to_{C^0} \equator\), we may choose \(T_{\reflectionangle} \in (0, T_S)\) such that \(d(M_t, \equator) < G_{\reflectionangle}/2\) for all \(t < -T_{\reflectionangle}\); that is, \(\radialdistance (x) < G_{\reflectionangle}/2\) for all \(x \in M_t\) and \(t < -T_{\reflectionangle}\). Now for \(x \in M_t^+ \intersect \radialprojection^{-1} \equator_{\mu}\), since \(\reflectionmap\) is an isometry, we have \(d(\reflectionmap(x), \reflectionmap(\radialprojection(x))) < G_{\reflectionangle}/2\); therefore, \(\radialdistance(\reflectionmap(x)) > G_{\reflectionangle}/2\). Consequently, on \(\reflectionset{\equator}^{+,\mu}\), we have \(\radialdistance(\reflectionmap(\reflectionset{(M_t)}^+)) > G_{\reflectionangle}/2\) and \(\radialdistance(M_t^-) > G_{\reflectionangle}/2\). In other words, given any \(\mu>0\), there is a \(T_{\reflectionangle}\) such that for all \(t \in (-T_S,T_{\reflectionangle})\),  we have
\begin{equation}
\label{eq:interiorestimate}
\reflectionmap(\reflectionset{(M_t)}^+) \geq \reflectionset{(M_t)}^- \quad \text{ over } \reflectionset{\equator}^{+,\mu}.
\end{equation}

\emph{Combined Estimates}
Choose \(T_{\reflectionangle}\) and \(\eta\) so that both the boundary estimate \cref{eq:boundaryestimate} holds. Now choose \(\mu\) so that \(\radialprojection(\reflectionmap(\reflectionset{\equator}^{+,\mu})) \union \{x \in \reflectionset{\equator}^- : d(x, \reflectionplane) = \reflectionset{\equator}^-\) which can be achieved since \(\reflectionmap\) is an isometry. Decreasing \(T_{\delta}\) if necessary ensures that the interior estimate \cref{eq:interiorestimate} also holds and hence,
\[
\reflectionmap(\reflectionset{(M_t)}^+) \geq \reflectionset{(M_t)}^-
\]
everywhere for all \(t \in (-T_S, -T_{\reflectionangle})\).
\end{proof}

\begin{lemma}
\label{lem:approximate_symmetrypreserved}
There exists a \(T\) independent of \(\reflectionangle\) such that for any \(\reflectionangle \in (0,\reflectionangle_0)\) we have \(\reflectionmap(\reflectionset{M_t^+}) \geq M_t^-\) for all \(t \in (-T_S, -T)\).
\end{lemma}
\begin{proof}
As remarked above, there exists a \(T\) such that both \(M_t\) and \(\reflectionmap(M_t\) are graphs over the equator on \((-T_S, T)\). Moreover, since both the equator \(\equator\) and the reflected equator \(\reflectionmap(\equator)\) meet \(\reflectionplane\) transversely, be decreasing \(T\) if necessary, we may assume \(M_t\) and \(\reflectionmap(M_t\) also meet \(\reflectionplane\) transversely. As with the monotonicity of the graph property in \(\reflectionangle\) we have monotonicity in transversal meeting and so \(T\) is independent of \(\reflectionangle\).

Recall the assumption that \(M_t\) satisfies a uniformly parabolic equation wherever it is strictly convex. Since \(\reflectionmap\) is an isometry, \(\reflectionmap(M_t)\) also is strictly convex and satisfies the same equation. Therefore \(f_t,g_t\) satisfy a uniformly parabolic equation on \((-T_S + \epsilon_{\reflectionangle}, -T)\) where \(0 < \epsilon_{\reflectionangle} < \min\{T_S-T, T_S-T_{\reflectionangle}\}\) is chosen so that we have a strict, positive lower bound on \(|A|\) from strict convexity and compactness and so that on \((-T_S + \epsilon_{\reflectionangle}, -T_{\reflectionangle})\) we have the condition \(\reflectionmap(\reflectionset{(M_t)}^+) \geq \reflectionset{(M_t)}^-\). That is, \(g_t \geq f_t\). We also have the boundary condition \(f_t = g_t\) on \(\reflectionplane\).

The comparison principle as in \cite[Theorem 3.1.1]{giga}, applied on the domain \(\union\limits_{t \in (-T_S + \epsilon_{\reflectionangle}, -T)} \radialprojection(\reflectionset(M_t)^-) \subseteq \equator\) yields that \(g_t \geq f_t\) holds forwards in time on all of \((-T_S + \epsilon_{\reflectionangle}, -T)\) hence on \((-T_S, -T)\). 
\end{proof}
\begin{theorem}
\label{thm:classification}
Let \(M_t\) be a convex, embedded (quasi-)ancient solution of any (uniform whenever on uniformly convex hypersurfaces) parabolic equation on \(\S^{n+1}\) such that \(M_t \to \equator\) in \(C^1\) as \(t \to -T_S\) with uniform \(C^2\) bounds. Then \(M_t\) is a family of shrinking geodesic spheres emanating from the equator \(\equator\) at \(t=-T_S\).
\end{theorem}
\begin{proof}
From \cref{lem:approximate_symmetrypreserved} we have \(\reflectionmap(\reflectionset{(M_t)}^+) \geq \reflectionset{(M_t)}^-\) everywhere for all \(t \in (-T_S, -T)\) and any \(\reflectionangle \in (0,\reflectionangle_0)\). By continuity therefore, sending \(\reflectionangle \to 0\) we have \(\reflectionmap(\reflectionset{(M_t)}^+) \geq \reflectionset{(M_t)}^-\) for all \(t \in (-T_S, -T)\)  and any \(\reflectionvector\) satisfying \(\ip{\reflectionvector}{\vertvec} = 0\).

Now, we need some simple properties of $\reflectionmap[\reflectionvector_0]$ following from the fact that $\ip{\reflectionvector}{\vertvec} = 0$:
\begin{itemize}
\item $\reflectionmap^2 = \id$,
\item $S \geq T \Rightarrow \reflectionmap(S) \geq  \reflectionmap(T)$,
\item $\reflectionmap = \reflectionmap[-\reflectionvector]$, and
\item $\reflectionset{S}^{\pm} = \reflectionset[-\reflectionvector]{S}^{\mp}$.
\end{itemize}
Thus we obtain,
\begin{align*}
\reflectionset{(M_t)}^+ &= \reflectionmap(\reflectionmap(\reflectionset{(M_t)}^+)) \geq \reflectionmap(\reflectionset{(M_t)}^-) \\
&= \reflectionmap[-\reflectionvector](\reflectionset[-\reflectionvector]{(M_t)^+}) \geq \reflectionset[-\reflectionvector]{(M_t)}^-\\
&= \reflectionset{(M_t)}^+.
\end{align*}
We must have equality all the way through and hence the middle line implies
\[
\reflectionmap[-\reflectionvector](\reflectionset[-\reflectionvector]{(M_t)^+}) = \reflectionset[-\reflectionvector]{(M_t)}^-
\]
for any $\reflectionvector$.

Thus \(M_t\) is invariant under \(\reflectionmap\) for any \(\reflectionvector\) satisfying \(\ip{\reflectionvector}{\vertvec} = 0\) and hence is a geodesic sphere for every \(t \in (-T_S, T)\) and by uniqueness of solutions, is a geodesic sphere for every \(t \in (-T_S, 0)\).
\end{proof}
\begin{example}
We consider a family of convex curves $\gamma_t$ on $\mathbb{S}^2$ that their spherical radial functions evolve by
\[\rho:\mathbb{S}^1\times[0,T)\to \mathbb{R}\]
\begin{equation}\label{eq: angenent oval}
\partial_t\rho(\cdot,t)=-\kappa \frac{\sqrt{\sin^2\rho+\rho_{\theta}^2}}{\sin\rho}\frac{\sin^2\rho+\rho_{\theta}^2}{\tan^2\rho+(\tan\rho)_{\theta}^2}(\cdot,t).
\end{equation}
Here, $\kappa(\cdot,t)$ is the curvature of the curve $\gamma_t$ with the radial function $\rho(\cdot,t).$ 
In the polar coordinates, we can express the curvature as follows:
\[\kappa=\frac{-\rho_{\theta\theta}\sin\rho+2\rho_\theta^2\cos\rho+\cos\rho\sin^2\rho}{(\sin^2\rho+\rho_{\theta}^2)^{\frac{3}{2}}}.\]
We will show that the gnomonic projection of $\gamma_t$, denoted by $\bar{\gamma}_t$, evolves by the curve shortening flow in $\mathbb{R}^2.$ Write $\bar{\rho}(\cdot,t)$ for the radial function of $\bar{\gamma}_t$. We recall from \cite[page 8]{besau2014spherical} that
$\bar{\rho}=\tan\rho.$ Using this formula and the expression of $\kappa$ we can write the curvature of $\bar{\gamma}_t$,  $\bar{\kappa}(\cdot,t),$ as follows:
\[\kappa=\left(\frac{\bar{\rho}^2+\bar{\rho}_{\theta}^2}{(1+\bar{\rho}^2)(\sin^2\rho+\rho_{\theta}^2)}\right)^{\frac{3}{2}}\bar{\kappa}=\left(\frac{\bar{\rho}^2+1}{\bar{h}^2+1}\right)^{\frac{3}{2}}\bar{\kappa}.\]
Here $\bar{h}=\frac{\bar{\rho}}{\sqrt{\bar{\rho}^2+\bar{\rho}_{\theta}^2}}$ is the support function of $\bar{\gamma}.$
Therefore, 
\begin{align*}
\partial_t\bar{\rho}&=-\bar{\kappa}(1+\bar{\rho}^2)\left(\frac{\bar{\rho}^2+\bar{\rho}_{\theta}^2}{(1+\bar{\rho}^2)(\sin^2\rho+\rho_{\theta}^2)}\right)^{\frac{3}{2}}\frac{\sqrt{\sin^2\rho+\rho_{\theta}^2}}{\sin\rho}\frac{\sin^2\rho+\rho_{\theta}^2}{\tan^2\rho+(\tan\rho)_{\theta}^2}\\
&=-\bar{\kappa}(1+\bar{\rho}^2)^{\frac{3}{2}}\left(\frac{\bar{\rho}^2+\bar{\rho}_{\theta}^2}{(1+\bar{\rho}^2)(\sin^2\rho+\rho_{\theta}^2)}\right)^{\frac{3}{2}}\frac{\sqrt{\sin^2\rho+\rho_{\theta}^2}}{\bar{\rho}}\frac{\sin^2\rho+\rho_{\theta}^2}{\tan^2\rho+(\tan\rho)_{\theta}^2}\\
&=-\bar{\kappa}\frac{\sqrt{\bar{\rho}^2+\bar{\rho}_{\theta}^2}}{\bar{\rho}}.
\end{align*}
The curve shortening flow in $\mathbb{R}^2$ has non-trivial ancient solutions, the Angenent ovals. Thus, there exists a non-trivial convex ancient solution to the flow (\ref{eq: angenent oval}). This ancient solution converges backwards in time to a lune. 
\end{example}
\begin{example}
We consider a family of convex curves $\gamma_t$ on $\mathbb{S}^2$ that their spherical radial functions evolve by
\[\rho:\mathbb{S}^1\times[0,T)\to \mathbb{R}\]
\begin{equation}\label{eq: ellipse}
\partial_t\tan\rho(\cdot,t)=-\kappa^{\frac{1}{3}} \frac{\sqrt{\sin^2\rho+\rho_{\theta}^2}}{\sin\rho}(\cdot,t).
\end{equation}
The gnomonic projection of $\gamma_t$ evolves by the affine normal flow in $\mathbb{R}^2:$ 
\begin{align*}
\partial_t\bar{\rho}&=-\bar{\kappa}^{\frac{1}{3}}\frac{\sqrt{\bar{\rho}^2+\bar{\rho}_{\theta}^2}}{\bar{\rho}}.
\end{align*}
Origin-centered ellipses are ancient solutions to the affine normal flow in $\mathbb{R}^2$. Thus, there exists a non-trivial convex ancient solution to the flow (\ref{eq: ellipse}). This ancient solution converges backwards in time to an equator. 
\end{example}
\bibliographystyle{amsplain}
\bibliography{Bibliography.bib}


\end{document}
